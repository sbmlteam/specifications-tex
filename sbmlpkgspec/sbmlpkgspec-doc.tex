\documentclass{sbmlpkgspec}

% Macros just for this document:

\newcommand{\sbmlpkg}{\textsc{SBMLPkgSpec}\xspace}
\newcommand{\sbmlpkgfile}{\texttt{sbmlpkgspec.cls}\xspace}
\newcommand{\latex}{\LaTeX{}\xspace}
\newcommand{\tex}{\TeX{}\xspace}
\newcommand{\distURL}{http://sourceforge.net/projects/sbml/files/specifications/tex}
\newcommand{\srcURL}{https://sbml.svn.sourceforge.net/svnroot/sbml/trunk/project/tex}
\newcommand{\webURL}{http://sbml.org/Documents/Specifications/LaTeX}
\newcommand{\cmd}[1]{\texttt{\textbackslash #1}}

\reversemarginpar  % Want "\watchout" to be put on the left, not the right.
\newcommand{\watchout}{\marginpar{\hspace*{34pt}\raisebox{-0.5ex}{\Large\ding{43}}}}


% Document body:

\begin{document}

\packageTitle{\latex Class for SBML Package Specifications}
\packageVersion{Version 1.0.1}
\packageVersionDate{5 October 2011}

\title{\sbmlpkg: A \latex Class for SBML Level~3 \mbox{Package Specification Documents}}

\author{Michael Hucka\\[0.25em]
  \mailto{mhucka@caltech.edu}\\[0.25em]
  Computing and Mathematical Sciences\\
  California Institute of Technology\\
  Pasadena, CA, USA
}

\maketitlepage
\maketableofcontents


% -----------------------------------------------------------------------------
\section{Introduction}
% -----------------------------------------------------------------------------

The \sbmlpkg document class for \latex provides a standard format for SBML
Level~3 \emph{package specification} documents.  In the paragraphs below, I
explain how to use this document class.  I assume readers are already
familiar with \latex.  (There are many tutorials available on \latex today,
and readers should have not trouble finding them on the Internet if they
need more information than what is provided here.)

The document before you is itself formatted using the \sbmlpkg class, with
a minor change to omit some SBML package-specific text normally placed on
the front page.  (For example, the front page of this document does not
announce it is an ``SBML Level 3 Package Specification'', nor does it
provide a package URL and other similar information.)  In all other
respects, ``what you see is what you get''---this is the appearance of an
SBML Level~3 package specification document when it is formatted with
\sbmlpkg.

Incidentally, in the rest of this document, I usually refer to the \sbmlpkg
class as simply ``\sbmlpkg''.



\subsection{Prerequisites and installation}

The \sbmlpkg class itself consists of only a single file, \sbmlpkgfile.  It
comes with some accompanying documentation (specifically, the file you are
reading, along with the source files to produce it), but the actual class
definition is in the file \sbmlpkgfile.  Its use should require only a
recent and relatively complete installation of \latex2e.  I developed and
tested \sbmlpkg with the TeX Live 2009 distribution on a Mac~OS~X 10.6.x
system.  \tab{where} lists the Internet locations from where you may obtain
\sbmlpkg.

\begin{table}[hb]
  \begin{edtable}{tabular}{ll}
    \toprule
    \textbf{Item} & \textbf{Location} \\
    \midrule
    Distribution archive & \url{\distURL}\\
    Web page		 & \url{\webURL}\\
    Source tree (SVN)    & \url{\srcURL}\\
    \bottomrule
  \end{edtable}
  \caption{Where to find \sbmlpkg on the Internet.}
  \label{where}
\end{table}


To use \sbmlpkg, you will need to inform your copy of \LaTeXe where to find
\sbmlpkgfile.  This can be done in a variety of ways.  Here are two common
ones:

\begin{itemize}

\item \emph{Per-document installation}.  This is arguably the simplest
  approach.  Download \sbmlpkg from a distribution site (see \tab{where}),
  copy the contents (specifically, \sbmlpkgfile and the folder named
  ``\texttt{logos}'') to the folder where you keep the other files for the
  SBML Level~3 package specification you are authoring, and you are done.
  The next time you run \latex in that folder (assuming you declare the
  document class as explained in \sect{usage}), it will find the
  \texttt{.cls} file in the current directory and be on its merry way.

\item \emph{``Central'' installation}.  In this approach, you install
  \sbmlpkgfile in a folder where you keep other \latex class files, and
  configure your copy of \latex to find things there.  Configuring a \tex
  system in this way on Unix-type systems (Linux, etc.)\ usually requires
  setting the environment variable \texttt{TEXINPUTS} and possibly others.
  Please consult the documentation for your \tex installation to determine
  how to do this.

\end{itemize}


\subsection{Special notation in this document}

Some paragraphs \watchout in this document include a hand pointer in the
left margin (illustrated at the left).  These are meant to call attention
to paragraphs containing significant points that may be too easily missed
during the first reading.  Readers may wish to revisit those paragraphs
once they are actually using \sbmlpkg in practice.


\subsection{Where to send bug reports and Feedback}

Please report problems you encounter with \sbmlpkg.  You can contact the
author directly, at the email address given on the cover page, or you can
file a bug report using the tracker at \url{http://sbml.org/issue-tracker}.


% -----------------------------------------------------------------------------
\section{Creating documents with \sbmlpkg}
\label{usage}
% -----------------------------------------------------------------------------

This section provides a summary of the main features and capabilities of
\sbmlpkg, and serves as a guide to getting started with this \latex class.


\subsection{Basic document structure}
\label{basic-structure}

The following fragment illustrates the basic structure of a simple input
file.

\begin{example}
\documentclass{sbmlpkgspec}
\begin{document}

\packageTitle{Example}
\packageVersion{Version 1 (Draft)}
\packageVersionDate{14 August 2011}
\packageGeneralURL{http://sbml.org/Documents/Specifications/Example}
\packageThisVersionURL{http://sbml.org/Documents/Specifications/Example_14_August_2011}

\author{Michael Hucka\\[0.25em]
  \mailto{mhucka@caltech.edu}\\[0.25em]
  Computing and Mathematical Sciences\\
  California Institute of Technology\\
  Pasadena, CA, USA
}

\maketitlepage
\maketableofcontents

\section{...}
...
\end{document}
\end{example}

The fragment above illustrates the general structure expected by \sbmlpkg.
First, several commands all beginning with the characters \cmd{package} set
various internal variables that are used by \sbmlpkg to produce the final
package specification document.  For example, there is a title for the
package (\cmd{packageTitle\{\}}), a version number for the package
(\cmd{packageVersion\{\}}), and more.  Next, the author is declared.  After
that, the commands \cmd{maketitlepage} and \cmd{maketableofcontents}
instruct \latex to produce a title page and table of contents,
respectively.  Then comes the real body of the document, with section
headings and so on.  Finally, the document is closed with the standard
\latex command \cmd{end\{document\}}.

If \watchout your document is a draft version, make sure to add the special
argument \texttt{[draftspec]} to the \cmd{documentclass} command.  This
causes the front page of your document to have the word ``DRAFT'' printed
on it in large gray type, and for every page footer to mention ``(DRAFT)''
in it.  Here is an example of how to use this option:

\begin{example}
\documentclass[draftspec]{sbmlpkgspec}
\begin{document}
...
\end{example}

Not shown here, but useful to know, is that \sbmlpkg provides an additional
command for putting a prominent notice on the front page.  Writing
\cmd{notice\{}\emph{text}\texttt{\}} will put \emph{text} in a red box in
the middle of the front page.

\sbmlpkg does not define special commands for formatting author
information beyond the \cmd{author} command (which is actually a standard
\latex command from the \texttt{article} document class).  It does,
however, provide the command \cmd{mailto}, which is designed to turn email
addresses into \texttt{mailto:} hyperlinks.  Any additional formatting of
author and affiliation information is up to you, using standard \latex
commands.  Of course, there are many times when multiple authors \emph{are}
involved, so it is useful to have an approach for handling that situation.
An easy approach is to embed a \texttt{tabular} environment inside the
\cmd{author} command.  The following is an example taken directly from an
actual SBML Level~3 package specification document:

\begin{example}
\author{%
  \begin{tabular}{c>{\hspace{20pt}}c}
    Lucian P. Smith                     & Michael Hucka\\[0.25em]
    \mailto{lpsmith@u.washington.edu}   & \mailto{mhucka@caltech.edu}\\[0.25em]
    Department of Bioengineering        & Computing and Mathematical Sciences\\
    University of Washington            & California Institute of Technology\\
    Seattle, WA, US                     & Pasadena, CA, US\\
  \end{tabular}
}
\end{example}

A \watchout final point about \sbmlpkg is worth mentioning right at the
outset.  When you run \latex (typically using the \texttt{pdflatex}
variant) and look at the output, you will often find that section
references, page references, and line numbers do not get refreshed
correctly after one invocation of the command.  With \sbmlpkg, you will
typically have to run the command not twice, but \emph{three} times, to get
the correct, final numbers, because it uses the \texttt{varioref} package.
This is typically not a problem during actual writing; the edit-run-preview
cycle is such that you usually only care to see the results of content
changes, and for that, running \latex only once is enough, even if it
leaves reference unadjusted.  However, when you \emph{are} interested in
checking (e.g.) figure and table references, then it is important to keep
in mind the fact that you need to run \latex three times in succession to
get all the reference updates to propagate through.  The rule of thumb is:
if you run \latex and the references look wrong, run it again.


\subsection{Tables and figures}
\label{about-tables}

\sbmlpkg preloads the \latex \texttt{graphicx} and \texttt{xcolor}
packages, giving authors immediate access to the functionality of these
extensions without having to include them manually.  For example, if you
had an illustration in a file named ``\texttt{example.pdf}'', the following
fragment would generate a simple figure containing it:

\begin{example}
\begin{figure}
  \includegraphics{example}
  \caption{The figure caption.}
  \label{fig:example}
\end{figure}
\end{example}

The \latex \texttt{graphicx} package is extremely powerful; it offers many
options for controlling the size/scale and other characteristics of
graphics files.  Please refer to its documentation for help on how to use
it fully.

To produce the most pleasing-looking documents, create your figures using
8~point Helvetica for the text font.  \sbmlpkg redefines figures and tables
to use Helvetica as the font family and an 8~point size by default, so
creating included figures with match characteristics will lead to more
consistent-looking documents.  The stylistic choices here were made not
only for aesthetic reasons; the tighter letter spacing of the sans serif
font, and the smaller font size, makes it easier for authors to fit
material into tables and figures.  The specific choice of Helvetica is also
driven in part by consideration of the tools available to authors.  In
particular, it is today common to find online drawing tools that offer
Helvetica as a font choice.

\sbmlpkg also preloads the \texttt{booktabs} package.  This provides
\cmd{toprule}, \cmd{midrule} and \cmd{bottomrule} (among others), which can
be used to produce attractive tables.  The following text is what produced
\fig{where}:

\begin{example}
\begin{table}[hb]
  \begin{edtable}{tabular}{ll}                 % From the lineno package; see text, Section 2.2
    \toprule                                   % From booktabs -- generates line at top
    \textbf{Item}        & \textbf{Location} \\
    \midrule                                   % From booktabs -- generates middle line
    Distribution archive & \url{\distURL} \\
    Web page             & \url{\webURL} \\
    Source tree (SVN)    & \url{\srcURL} \\
    \bottomrule                                % From booktabs -- generates line at bottom
  \end{edtable}
  \caption{Where to find \sbmlpkg on the Internet.}
  \label{where}
\end{table}
\end{example}

The table and figure environments in \sbmlpkg have been redefined to place
their content inside a \latex \texttt{centering} environment.  This causes
the content to be centered on the page, and consequently, you do not need
to add centering commands to your floats.

To refer to figures, tables and other elements in the document, please use
the special commands listed in \tab{ref-commands} instead of writing the
usual idioms ``Figure\textasciitilde\textbackslash ref\{...\}''.  The
commands in \tab{ref-commands} will not only produce content and page
references that are automatically hyperlinked to the appropriate locations
in the finished document; they will also take care of adding the tie for
you, and they use the package \texttt{varioref} instead of the regular
\latex \texttt{ref} command to produce more informative references to
figures and tables.

\begin{table}[h]
  \begin{edtable}{tabular}{lll}
    \toprule
    \textbf{Command}				 & \textbf{Purpose} & \textbf{Output}\\
    \midrule
    \texttt{\textbackslash fig\{}\emph{label\,}\texttt{\}}  & Figure reference	& Figure X\\
    \texttt{\textbackslash tab\{}\emph{label\,}\texttt{\}}  & Table reference	& Table X\\
    \texttt{\textbackslash sect\{}\emph{label\,}\texttt{\}} & Section reference	& Section X\\
    \bottomrule
  \end{edtable}
  \caption{Commands for referring to figures and other entities in an
    \sbmlpkg document.  Use the commands with an argument consisting of the
    label being referenced.  For example: \cmd{fig\{myfig\}}.}
  \label{ref-commands}
\end{table}

Finally, in the case of long tables, readability is often enhanced by
adding a background color to every other row.  Once again, \sbmlpkg
preloads a \latex package (in this case, \texttt{xcolor} with the
\texttt{[table]} option) that provides a convenient facility for
automatically coloring alternate rows in a table.  Although many variations
are possible, for consistency between SBML package specification documents,
I recommend using one in particular:

\begin{example}
\rowcolors{2}{sbmlrowgray}{}
\end{example}

Simply insert the literal line above after the opening \cmd{begin\{table\}}
of your table, and proceed as usual.  The resulting table will have a light
gray background color for every other row.  This is demonstrated in
\fig{sbmlcore}, which was produced using the following opening sequence:

\begin{example}
\begin{table}[hbt]
  \rowcolors{2}{sbmlrowgray}{}
  \begin{edtable}{tabular}{ll}
  ...
\end{example}
%\end{table} % To avoid confusing XEmacs's paragraph filling algorithm
  
Note \watchout that tables are \emph{not} defined by \sbmlpkg to use
alternate-row background coloring by default, because in some situations
(such as short tables, or tables containing color), it is unnecessary and
distracting.  You must add the \cmd{rowcolors} manually, when it's
appropriate.


\subsection{Hyperlinks}
\label{hyperlinks}

In the last example in the previous section, you may have noticed the use
of the command \cmd{url\{\}}.  This command comes as part of the \latex
\texttt{hyperref} package, and like the other packages mentioned in this
section, it is also preloaded by \sbmlpkg.  It provides a number of
facilities that are used to implement features of \sbmlpkg.  \tab{hyperref}
lists some commands you may find useful in writing SBML specification
documents.

\begin{table}[h]
  \begin{edtable}{tabular}{lll}
    \toprule
    \textbf{Command}		 & \textbf{Purpose} \\
    \midrule
    \cmd{url\{}\emph{URL\,}\texttt{\}}
    &  Produce a hyperlinked reference to \emph{URL}\\
    \cmd{nolinkurl\{}\emph{URL\,}\texttt{\}}
    & Format \emph{URL} in the same way as \texttt{\textbackslash url}, without making it a hyperlink\\
    \cmd{href\{}\emph{URL\,}\texttt{\}\{}\emph{text}\texttt{\}} 
    & Make \emph{text} a hyperlink to \emph{URL}\\
    \bottomrule
  \end{edtable}
  \caption{Commands provided by \texttt{hyperref} for creating hyperlinks.}
  \label{hyperref}
\end{table}


\subsection{Examples and literal text}
\label{example-env}

A document about file formats and programming often includes passages meant
to be literal text.  Conventionally, these are typeset in a monospaced
type face resembling the output of typewriter.  There are two ways of
accomplishing this for SBML package specifications.  The first is to use
\latex's standard \mbox{\texttt{\textbackslash
    texttt\{}\emph{text}\texttt{\}}} command.  This causes ``\emph{text}''
to be output in a fixed-width font, like so: ``\texttt{text}''.

The second method is to use an environment defined by \sbmlpkg and
implemented using the \latex package \texttt{listings}.  This environment
is called \texttt{example}.  Wrapping any text with \cmd{begin\{example\}}
and \cmd{end\{example\}} will cause the text to be output by itself with a
gray box behind it, as in this example:

\begin{example}
This is an example of content placed within an "example" environment.
\end{example}

The \texttt{example} environment is particularly powerful.  Anything placed
inside it will be taken literally---even \latex commands will be ignored,
except for \cmd{end\{example\}}.  In fact, the entire contents of the
document fragment shown in \sect{basic-structure}, including the
\cmd{begin\{document\}} and \cmd{end\{document\}}, were all left unchanged
within the \texttt{example} environment used to produce the example.

Of course, sometimes you \emph{do} want an embedded \latex command to be
interpreted inside the \texttt{example} environment.  For those situations,
surround the \latex sequence with vertical bar (\texttt{|}) characters.
The vertical bar is defined by \sbmlpkg as the escape character for the
\texttt{example} environment. 

Finally, for those cases when it is more convenient to put the example
contents in a separate file, \sbmlpkg provides the command
\mbox{\cmd{exampleFile\{}\emph{file}\texttt{\}}}.  It will insert the
contents of \emph{file} at the point where it is invoked in the text, and
format the contents in the same style as the \texttt{example} environment.


\subsection{Color}

To help increase the style consistency between SBML specification
documents, \sbmlpkg defines a number of custom color names that authors may
use with commands such as \texttt{color}, \texttt{colorbox}, etc.
\tab{colors} provides a list of these color names. 

\newcommand{\drawcol}[1]{\colorbox{#1}{\color{#1}\rule{10ex}{1ex}}}

\begin{table}[hbt]
  \begin{edtable}{tabular}{lllll}
    \toprule
    \textbf{Color name}		& \textbf{Color sample} 	& \multicolumn{3}{c}{\textbf{RGB color value}} \\
    \midrule
    \texttt{sbmlblue}		& \drawcol{sbmlblue}		& red = 0.08	& blue = 0.51	& green = 0.77 \\
    \texttt{sbmlgray}		& \drawcol{sbmlgray}		& red = 0.7	& blue = 0.7	& green = 0.7 \\
    \texttt{sbmlrowgray}	& \drawcol{sbmlrowgray}		& red = 0.94	& blue = 0.94	& green = 0.94 \\
    \texttt{sbmlchangedcolor}	& \drawcol{sbmlchangedcolor}	& red = 0.69	& blue = 0.19	& green = 0.376 \\
    \texttt{extremelylightgray}	& \drawcol{extremelylightgray}	& red = 0.97	& blue = 0.97	& green = 0.97 \\
    \texttt{verylightgray}	& \drawcol{verylightgray}	& red = 0.9	& blue = 0.9	& green = 0.9 \\
    \texttt{lightgray}		& \drawcol{lightgray}		& red = 0.8	& blue = 0.8	& green = 0.8 \\
    \texttt{mediumgray}		& \drawcol{mediumgray}		& red = 0.5	& blue = 0.5	& green = 0.5 \\
    \texttt{darkgray}		& \drawcol{darkgray}		& red = 0.3	& blue = 0.3	& green = 0.3 \\
    \texttt{almostblack}	& \drawcol{almostblack}		& red = 0.2	& blue = 0.2	& green = 0.2 \\
    \texttt{darkblue}		& \drawcol{darkblue}		& red = 0.1	& blue = 0.4	& green = 0.55 \\
    \texttt{lightyellow}	& \drawcol{lightyellow}		& red = 0.995	& blue = 0.97	& green = 0.93 \\
    \texttt{mediumgreen}	& \drawcol{mediumgreen}		& red = 0.1	& blue = 0.6	& green = 0.3 \\
    \bottomrule
  \end{edtable}
  \caption{Color names defined by \sbmlpkg.  These names may be used in
    addition to the names of colors defined by the \latex package
    \texttt{xcolor}.}
  \label{colors}
\end{table}

Some of the colors in \tab{colors} are used by \sbmlpkg itself, for some of
the design elements such as section dividers and background colors.  Others
are colors that sometimes prove useful in different contexts of a
specification document, and still others are colors that were used in past
SBML documentation and are carried over in case they prove useful again in
the future.

When a specification document is a revision of a previous document, it is
the convention in SBML documentation to indicate text changes by coloring
them in red.  For this purpose, \sbmlpkg defines a command and an
environment.  The command \mbox{\texttt{\textbackslash
    changed\{}\emph{text}\texttt{\}}} causes \emph{text} to be written in
red (or more precisely, the color defined by \cmd{sbmlchangedcolor}, which
is a dark purple red), \changed{like this}.  It can be used for long
stretches of text and include embedded spaces and formatting commands.
Alternatively, for coloring even longer stretches of text and multiple
paragraphs, you may prefer to use the environment \texttt{blockChanged}.


\subsection{Commands for SBML constructs}

SBML defines a number of commands for referring to objects defined in the
main SBML specifications, such as \Species, \Reaction, etc.  Within a given
main SBML specification document---which are all in PDF format---the
mention of an object name is hyperlinked to the definition of that object
elsewhere in the document, such that clicking on the name causes the PDF
reading program to jump to the definition in the file.  Unfortunately,
while it is technically possible to hyperlink from SBML package documents
to specific parts within an external PDF file located somewhere on the
Internet, the result is more confusing and annoying than helpful.

\sbmlpkg therefore defines two sets of commands: one set predefines each of
the SBML Level~3 Core object names, but without hyperlinking, and the
second set lets package authors define new object names, with hyperlinking.
The result is that references to Core SBML objects are displayed in black
and without linking, while package objects appear in blue, as hyperlinks.
The linking is implemented using the standard \latex \texttt{hyperref}
package.

In addition to these commands, \sbmlpkg provides other commands for
typesetting primitive type names (such as \primtype{SId},
\primtype{double}, and so on) and other entities and XML fragments.  The
need for using these occurs routinely when writing SBML specifications.
The commands are described in \sect{primtype} and \sect{formatting-xml}.


\subsubsection{Predefined SBML Core object and type names}
\label{predefined-classes}

\tab{sbmlcore} lists the commands to typeset the names of SBML Level~3 Core
objects.  They are designed to be as convenient to use as possible,
requiring only one additional character (i.e., the leading backslash
character) to be typed beyond the name of the object itself.

\begin{table}[htb]
  \rowcolors{2}{sbmlrowgray}{}
  \renewcommand{\arraystretch}{0.95}
  \begin{edtable}{tabular}{ll}
    \toprule
    \textbf{Command}                      & \textbf{Object} \\
    \midrule
    \cmd{AlgebraicRule}                   & \AlgebraicRule \\
    \cmd{Annotation}                      & \Annotation \\
    \cmd{AssignmentRule}                  & \AssignmentRule \\
    \cmd{Compartment}                     & \Compartment \\
    \cmd{Constraint}                      & \Constraint \\
    \cmd{Delay}                           & \Delay \\
    \cmd{EventAssignment}                 & \EventAssignment \\
    \cmd{Event}                           & \Event \\
    \cmd{FunctionDefinition}              & \FunctionDefinition \\
    \cmd{InitialAssignment}               & \InitialAssignment \\
    \cmd{KineticLaw}                      & \KineticLaw \\
    \cmd{ListOfCompartments}              & \ListOfCompartments \\
    \cmd{ListOfConstraints}               & \ListOfConstraints \\
    \cmd{ListOfEventAssignments}          & \ListOfEventAssignments \\
    \cmd{ListOfEvents}                    & \ListOfEvents \\
    \cmd{ListOfFunctionDefinitions}       & \ListOfFunctionDefinitions \\
    \cmd{ListOfInitialAssignments}        & \ListOfInitialAssignments \\
    \cmd{ListOfLocalParameters}           & \ListOfLocalParameters \\
    \cmd{ListOfModifierSpeciesReferences} & \ListOfModifierSpeciesReferences \\
    \cmd{ListOfPackages}                  & \ListOfPackages \\
    \cmd{ListOfParameters}                & \ListOfParameters \\
    \cmd{ListOfReactions}                 & \ListOfReactions \\
    \cmd{ListOfRules}                     & \ListOfRules \\
    \cmd{ListOfSpeciesReferences}         & \ListOfSpeciesReferences \\
    \cmd{ListOfSpecies}                   & \ListOfSpecies \\
    \cmd{ListOfUnitDefinitions}           & \ListOfUnitDefinitions \\
    \cmd{ListOfUnits}                     & \ListOfUnits \\
    \cmd{LocalParameter}                  & \LocalParameter \\
    \cmd{Message}                         & \Message \\
    \cmd{Model}                           & \Model \\
    \cmd{ModifierSpeciesReference}        & \ModifierSpeciesReference \\
    \cmd{Notes}                           & \Notes \\
    \cmd{Package}                         & \Package \\
    \cmd{Parameter}                       & \Parameter \\
    \cmd{Priority}                        & \Priority \\
    \cmd{RateRule}                        & \RateRule \\
    \cmd{Reaction}                        & \Reaction \\
    \cmd{Rule}                            & \Rule \\
    \cmd{SBML}                            & \SBML \\
    \cmd{SBase}                           & \SBase \\
    \cmd{SimpleSpeciesReference}          & \SimpleSpeciesReference \\
    \cmd{SpeciesReference}                & \SpeciesReference \\
    \cmd{Species}                         & \Species \\
    \cmd{StoichiometryMath}               & \StoichiometryMath \\
    \cmd{Trigger}                         & \Trigger \\
    \cmd{UnitDefinition}                  & \UnitDefinition \\
    \cmd{Unit}                            & \Unit \\
    \bottomrule
  \end{edtable}
  \caption{Commands for the names of object classes defined in the SBML Level~3
    Core specification.} 
  \label{sbmlcore}
\end{table}

\subsubsection{Defining new object names and types in package specifications}
\label{defining-classes}

When an SBML package specification document defines new object classes, it
is useful to make all mentions of the class name be hyperlinks to the class
definition in the document.  To this end, \sbmlpkg provides commands that
can be used to create new \latex commands to define hyperlinked name
references.  The purpose of these commands is to let package authors define
commands of the form \cmd{ObjectName} that print the name
\texttt{ObjectName} and simultaneously make it a hyperlink to the sections
in the document where \textsf{\textbf{ObjectName}} is defined.  They are
best used conjunction with \latex's \cmd{newcommand} command:

\begin{description}[font=\normalfont\ttfamily,style=nextline]

\item[\textbackslash defRef\{\emph{name}\}\{\emph{section}\}] Create a
  hyperlinked reference to the section labeled \emph{section} and call it
  \emph{name}.  The reference appears at the point in the text where the
  \texttt{\textbackslash defRef} command is invoked.  This command is
  intended for references to regular (not abstract) classes; see the next
  command for abstract classes.

\item[\textbackslash absDefRef\{\emph{name}\}\{\emph{section}\}] Create a
  hyperlinked reference to the section labeled \emph{section} and call it
  \emph{name}.  The reference appears at the point in the text where the
  \texttt{\textbackslash defRef} command is invoked.  This command is
  intended for references to abstract classes; see the previous command for
  the corresponding command for non-abstract classes.

\end{description}

The following is an example of how the commands above may be used; this is
taken straight from the source files of the SBML Level~3 Version~1 Core
specification document:

\begin{example}
\newcommand{\SBase}{\absDefRef{SBase}{sec:sbase}\xspace}
\newcommand{\SBML} {\defRef{SBML}{sec:sbml}\xspace}
\newcommand{\Model}{\defRef{Model}{sec:model}\xspace}
\end{example}

The section labels ``\texttt{sec:base}'', ``\texttt{sec:sbml}'', etc., are
defined using \latex's \cmd{label} command at the beginning of each section
where the corresponding objects are defined.  (The command \cmd{xspace} is
discussed in \sect{included-classes}.)

Sometimes it is desirable to write the names of object classes without
introducing hyperlinks.  A common situation is when mentioning the names of
the object classes or types in section headings.  In these situations,
instead of using the \cmd{ObjectName} commands, you may use the following
commands provided by \sbmlpkg:

\begin{description}[font=\normalfont\ttfamily,style=nextline]

\item[\textbackslash class\{\emph{name}\}] Typesets
  \emph{name} in the same font style as used by \texttt{\textbackslash
    defRef\{}\emph{name}\texttt{\}\{}\emph{section}\texttt{\}}, without
  a hyperlink.

\item[\textbackslash abstractclass\{\emph{name}\}] Typesets
  \emph{name} in the same font style as used by \texttt{\textbackslash
    absDefRef\{}\emph{name}\texttt{\}\{}\emph{section}\texttt{\}}, without
  a hyperlink.

\end{description}



\subsubsection{Commands for formatting the names of primitive data types}
\label{primtype}

A convention that has evolved over the years of writing SBML specifications
is to typeset the names of primitive data types (such as \primtype{SId}) in
a monospaced, typewriter-like type face, without hyperlinks.  SBML packages
may define their own new primitive types.  To format the names of these
types in a style consistent with the SBML Level~3 Core specification
document, package authors should use the \cmd{primtype} and
\cmd{primtypeNC} commands:

\begin{description}[font=\normalfont\ttfamily,style=nextline]

\item[\textbackslash primtype\{\emph{name}\}] Typesets
  the name of a primitive data type \emph{name}.  SBML Level~3 Core also
  defines and uses a number of primitive data types, but these do not have
  separate commands in \sbmlpkg.  Instead, they should be written using the
  command \mbox{\texttt{\textbackslash primtype\{}\emph{type}\texttt{\}}},
  where \emph{type} is one of the following names:

  \begin{center}\ttfamily
    \begin{edtable}{tabular}{llll}
      boolean	& int		 	& SId	 	& UnitSId\\
      double	& positiveInteger	& SIdRef	& UnitSIdRef\\
      ID	& SBOTerm		& string \\
    \end{edtable}
  \end{center}

\item[\texttt{\textbackslash primtypeNC\{}\emph{name}\texttt{\}}] Like
  \texttt{\textbackslash primtype}, but does not force the color of the
  text to be black.  The main use of this variant of the command is when
  writing the names of primitive types in the arguments to \latex
  sectioning commands (e.g., \cmd{section}, \cmd{subsection} and the like),
  to avoid the change in color that would otherwise occur in the document's
  table of contents.  (The \cmd{primtype} command sets the color of the
  text to pure black, to make the text stand out more in the document body.
  The change of color would occur because the entries in the table of
  contents are all hyperlinks to the beginning of the sections, and
  hyperlinks are colored blue.)

\end{description}

For example, the command sequence ``\mbox{\texttt{\textbackslash
    primtype\{SId\}}}'' written somewhere in the body of a document
produces ``\primtype{SId}'' in the formatted output.   On the other hand,
the following illustrates how to write the name of a type in a section
heading: 

\begin{example}
\subsection{This is the documentation for \primtypeNC{MySpecialSId}}
\end{example}


\subsubsection{Commands for formatting in-text XML descriptions}
\label{formatting-xml}

In addition to formatting the names of SBML object classes and primitive
data types, \sbmlpkg provides commands for formatting text meant to be
literal examples of XML.  These special commands are provided so in-text
descriptions of XML constructs can be formatted in a way and attractive
way:

\begin{description}[font=\normalfont\ttfamily,style=nextline]

\item[\textbackslash token\{\emph{text}\}] Formats literal XML tokens, such
  as attribute names.  Do not use this to format text with embedded spaces;
  instead, use multiple \texttt{\textbackslash token} commands separated by
  spaces.

\item[\textbackslash val\{\emph{text}\}] Format the value of an attribute.
  This is essentially \cmd{token} but surrounded by double quotes.

\item[\textbackslash uri\{\emph{text}\}] Format a URI.  This is essentially
  \cmd{token} but surrounded by double quotes; it's provided to help make
  the formatting of URIs in documents more consistent.

\end{description}

Here is an example of using the commands above.  Suppose you wanted to
write that a certain attribute named \texttt{attrib1} is to be assigned the
value \val{10}.  The following input text,

\begin{example}
\token{attrib1}=\val{10}
\end{example}

will result in the following formatted output:

\begin{center}
\token{attrib1}=\val{10}
\end{center}

Note that the command \cmd{token} and others above are only meant to be
used for single in-text mentions of tokens and XML attribute-value pairs.
They are not suitable for longer content; for those cases, use the
\texttt{example} environment described in \sect{example-env}.


\subsection{Line numbers}
\label{about-lineno}

\sbmlpkg preloads the \latex package \texttt{lineno} and configures it to
produce the line numbers that you see in the right column of every page.
These line numbers are important for specification documents because they
allow discussion and bug reports to refer to specific portions of the text
more easily.  Crucially, lists of errata and other issues recognized
\emph{after} a specification is released need ways of referring to precise
locations in the specification document, and line numbers are invaluable
for that purpose.

For \watchout the most part, you needn't do anything to get line numbers to
be added to your document.  There are exceptions: certain content such as
tabular material and floats are not handled by \texttt{lineno} very well,
and require manual intervention.  Some cases cannot be fixed at all,
notably figures incorporated from external files, but tabular material is
fixable.  To get line numbers to be displayed in tables, you will need to
wrap all uses of \texttt{tabular} with the special environment
\texttt{edtable}.  The basic idiom is the following:

\begin{example}
\begin{edtable}{tabular}{|\color{gray}\emph{...normal tabular column specifiers...}\color{\normaltextcolor}|}
  |\color{gray}\emph{...tabular content...}\color{\normaltextcolor}|
\end{edtable}
\end{example}

The \texttt{edtable} environment is able to wrap a number of standard
\latex environments, of which \texttt{tabular} is probably the most useful
for most kinds of tables.  Practical examples of using \texttt{edtable}
appear below in other sections.


\subsection{SBML validation rules}
\label{validation-rules}

A convention developed while writing the core SBML specification documents
is to define validation and consistency rules that must or should be
satisfied in order for a document to conform to the specification.  SBML
package specifications should likewise follow this convetion and define
their own validation and consistency rules.  The convention identifies
different degrees of rule strictness.  Formally, the differences are
expressed in the statement of a rule: either a rule states that a condition
\emph{must} be true, or a rule states that it \emph{should} be true.  Rules
of the former kind are strict SBML validation rules---a model encoded in
SBML must conform to all of them in order to be considered valid.  Rules of
the latter kind are consistency rules.

To help highlight these differences, the SBML specification documents and
\sbmlpkg provide commands to format the three kinds of rules, with three
different symbols:

\begin{description}

\item[\hspace*{6.5pt}\vSymbol\vsp] A \vSymbolName indicates a
  \emph{requirement} for SBML conformance. If a model does not follow this
  rule, it does not conform to the specification.  (Mnemonic intention
  behind the choice of symbol: ``This must be checked.'')

\item[\hspace*{6.5pt}\cSymbol\csp] A \cSymbolName indicates a
  \emph{recommendation} for model consistency.  If a model does not follow
  this rule, it is not considered strictly invalid as far as the
  specification is concerned; however, it indicates that the model contains
  a physical or conceptual inconsistency.  (Mnemonic intention behind the
  choice of symbol: ``This is a cause for warning.'')

\item[\hspace*{6.5pt}\mSymbol\msp] A \mSymbolName indicates a strong
  recommendation for good modeling practice.  This rule is not
  strictly a matter of SBML encoding, but the recommendation comes
  from logical reasoning.  As in the previous case, if a model does
  not follow this rule, it is not strictly considered an invalid SBML
  encoding.  (Mnemonic intention behind the choice of symbol: ``You're
  a star if you heed this.'')

\end{description}

\sbmlpkg defines three commands for writing these rules in SBML package
specifications documents:

\begin{description}[font=\normalfont\ttfamily,style=nextline]

\item[\cmd{validRule\{\emph{number}\}\{\emph{text}\}}] Format \emph{number}
  as a validation rule with the description \emph{text}.

\item[\cmd{consistencyRule\{\emph{number}\}\{\emph{text}\}}] Format \emph{number}
  as a consistency rule with the description \emph{text}.

\item[\cmd{modelingRule\{\emph{number}\}\{\emph{text}\}}] Format \emph{number}
  as a modeling rule with the description \emph{text}.

\end{description}



% -----------------------------------------------------------------------------
% \section{Additional conventions for SBML specification documents}
% \label{conventions}
% -----------------------------------------------------------------------------



% -----------------------------------------------------------------------------
% \section{What \sbmlpkg provides: a concise reference}
\section{Additional features of \sbmlpkg}
\label{provides}
% -----------------------------------------------------------------------------

% This section describes the commands and features provided by \sbmlpkg.

This section describes some additional aspects of \sbmlpkg.


\subsection{Options understood by \sbmlpkg}
\label{pkg-options}

A number of options may be given to \sbmlpkg in the \cmd{documentclass}
command that begins a document.  Here follows a complete list:

\begin{description}[font=\normalfont\ttfamily,style=nextline]

\item[\texttt{draftspec}] This option causes the front page of the document
  to contain the word ``DRAFT'' in large gray letters, and the footer of
  every page of the document to contain the word ``(DRAFT)''.  Authors should
  use this option until such time as the specification document is considered
  a release candidate or a final release.

\item[finalspec] (Default) This option causes the large ``DRAFT'' on the
  front page and ``(DRAFT)'' in the footers to be omitted.  It is the
  opposite of the \texttt{draftspec} option.

\item[toc] (Default) This option causes \sbmlpkg to include a table of
  contents as the second page of the document.  Whether the table has one
  wide column or two narrow columns is then controlled by the options
  \texttt{twocolumntoc} and \texttt{singlecolumntoc}, described below.

\item[notoc] This option causes the table of contents to be omitted.  (It
  is unclear under what circumstances one would want to omit the table of
  contents, but since some other \latex classes include this feature,
  \sbmlpkg follows suit.)

\item[twocolumntoc] This option causes \sbmlpkg to produce a two-column
  table of contents rather than the default one-column version.  (That is,
  unless the \texttt{notoc} option is also given, in which case, no table of
  contents is produced.)  This is useful when a document is long, with many
  sections, because the two-column version is more compact.  (However, beware
  that since the columns are narrower than the single-column version, long
  section names may become wrapped, leading to an aesthetically less pleasing
  result.  If you need to use two column output, you may alway want to
  examine whether you can shorten your section titles.)

\item[singlecolumntoc] (Default) This option is the opposite of
  \texttt{twocolumntoc}; it causes \sbmlpkg to produce a single column table
  of contents as the second page of the document.  (Again, if \texttt{notoc}
  is also given, then no table of contents is produced at all.)

\end{description}



% \subsection{Commands defined by \sbmlpkg}
% \label{cmds-defined}

% notice
% address
% package* commands


\subsection{Notable \latex packages preloaded by \sbmlpkg}
\label{included-classes}

As mentioned above, \sbmlpkg preloads many common \latex packages.  Some
are used to implement the features of \sbmlpkg itself; others are preloaded
so that authors do not have to load them explicitly.  Knowing what \sbmlpkg
provides upfront also makes it easier to explain how to write SBML
specification documents.

The following is a list of the \latex packages preloaded by \sbmlpkg.  It
is beyond the scope of this document to explain their features and
capabilities in detail; readers are urged to consult the documentation for
each one to learn more about them.  Documentation is available at CTAN
(\url{http://www.tug.org/ctan.html}).


\begin{itemize}

\item \texttt{amssymb}: This package defines many symbols and special
  characters.  In \sbmlpkg, it is used to get the symbols defined by the
  validation rule commands \cmd{validRule}, \cmd{consistencyRule} and
  \cmd{modelingRule} described in \sect{validation-rules}.

\item \texttt{array}: This provides a new and extended implementation of
  the \latex \texttt{array} environment and \texttt{tabular}
  environments.  It is particularly useful for the column formatting
  options it provides for the \texttt{tabular} environment. 

\item \texttt{booktabs}: Mentioned in \sect{about-tables}, this \latex
  package defines the commands \cmd{toprule}, \cmd{bottomrule}, and
  \cmd{midrule} for produced more attractive and professional-looking
  tables.

\item \texttt{enumitem}: This relatively new package adds facilities for more
  easily adjusting the look of \latex list environments, including the
  \texttt{description} environment---something that is difficult to do in
  plain \latex.  (Oh sure, there \emph{are} variables and commands for
  doing it in \latex, but they are limited and sometimes produce unexpected
  consequences elsewhere in a document.)

\item \texttt{graphicx}: A powerful and rich system for working with
  external graphics files in \latex documents. 

\item \texttt{hyperref}: Mentioned in \sect{hyperlinks}, this package defines
  commands for creating hyperlinks within and between documents.  \sbmlpkg
  uses it to define commands such as \cmd{fig} and \cmd{tab}.

\item \texttt{lineno}: Used by \sbmlpkg to implement line numbers on the
  page.  As mentioned in \sect{about-lineno}, not all line numbering can be
  accomplished completely without user intervention; in some cases, authors
  must add special commands to get line numbers into content on the page.

\item \texttt{listings}: Used by \sbmlpkg to implement the \texttt{example}
  environment and \cmd{exampleFile} command.

\item \texttt{multicol}: This provides an environment for putting multiple
  columns of text on a page.  In \sbmlpkg, it is used to produce two-column
  table of contents when the \texttt{[twocolumntoc]} option is given as
  described in \sect{pkg-options}.

\item \texttt{natbib}: This reimplements the normal \latex \cmd{cite}
  command to work with author-year citations and add a number of useful
  features.  It provides variants of \cmd{cite} such as \cmd{citep} and
  \cmd{citeauthor}.  The features of \texttt{natbib} are worth learning and
  using.

\item \texttt{varioref}: This defines commands such as \cmd{vref}, which is
  similar to \cmd{ref} but includes a page reference such as ``on the next
  page'' or ``on page 27'' when the corresponding \cmd{label} is not on the
  same page.  (The way that the page references are generated requires
  multiple passes of running \latex, which is one of the reasons why using
  \sbmlpkg requires 2--3 runs of \latex to generate final
  cross-references.)

\item \texttt{wasysym}: Similar to \texttt{amssymb}, this \latex package
  defines another set of symbols and special characters.  

\item \texttt{xcolor}: This package defines a large number of color names and
  associated commands.

\item \texttt{xspace}: A marvelous little extension that provides just one
  very useful command, \cmd{xspace}.  The \cmd{xspace} command can be used
  at the end of a macro designed to be expanded into text; it adds a space
  unless the macro is followed by a punctuation symbol.  This makes it
  possible to invoke the macro in text without adding an empty
  \texttt{\{\}} pair after it---something that would otherwise be a problem
  because \latex would consume the space character following the macro
  invocation, leading to a missing space in the final output.  The commands
  such as \cmd{SBML} are defined using \cmd{xspace}, so that in running
  text, you can write ``one two \cmd{SBML} three four'' and it will produce
  ``one two \SBML three four'' rather than ``one two \class{SBML}three
  four''.

\end{itemize}







% \subsection{Other potentially useful \latex classes \emph{not} preloaded by \sbmlpkg}

% There are \emph{many} \latex classes available in the world.  It would be
% impossible to do them all justice here, but authors may find some of the
% following useful in the current context of SBML specification documents.

% tikz
% mhchem



\section{Acknowledgments}

This work was made possible by grant R01 GM070923 from the NIH National
Institute of General Medical Sciences (USA) for continued development and
support of SBML and related software infrastructure.


\clearpage
\bibliography{sbmlpkgspec}



\end{document}
