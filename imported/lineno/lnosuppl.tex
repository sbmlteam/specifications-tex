%% LNoSuppl.tex -- U.L. 
\title{\LARGE \bf Supplementary files belonging \\ 
       to the \textit{lineno.sty} distribution \\[1ex]
       \normalfont \Large
       Lazy \texttt{ASCII}$\to$\texttt{PDF} listings}
\author{Uwe L\"uck}
\documentclass[10pt]{article}
\usepackage{verbatim}
\pagestyle{headings}
\newcommand*{\cs}[1]{\texttt{\char"5C#1}}
\newcommand*{\lcurl}[1]{\[\texttt{CTAN:/macros/latex/contrib/#1}\]}
\makeatletter
\newcommand*{\notinaux}[1]{% 
  \ifx\protect\@typeset@protect #1% 
  \else 
    \ifdim\lastskip>\z@\ignorespaces\fi 
  \fi}
\makeatother 
\begin{document}
\maketitle
\section*{Preface}

\texttt{lineno.sty} is a macro package made by 
Stephan~I.~B\"ottcher for attaching line numbers to 
\LaTeX\ documents. Some people have used it for revising 
submittings in collaboration with referees or co-authors. 
Documentations are nowadays preferred to be in 
Adobe's \texttt{PDF}---so \texttt{lineno.sty}'s 
documentation is \lcurl{lineno/lineno.pdf\,.}

\texttt{ednotes.sty} uses \texttt{lineno.sty} for critical 
editions, combining it with Alexander~I.~Rozhenko's 
\texttt{manyfoot.sty}---this was Christian Tapp's idea, 
who then hired me for adding the \TeX nical details. 
In doing this, I had to change some internals of 
\texttt{lineno.sty}, so Stephan transferred maintenance 
to me; then some of my macro files that I originally had 
made for \texttt{ednotes.sty} wandered into the 
\texttt{lineno} directory of CTAN---because they turned 
out not to need \texttt{ednotes.sty}, 
just to work as extensions of \texttt{lineno.sty}\,. 

Now, I haven't had the time for making \texttt{.dtx} versions 
of the \texttt{.sty} files for \texttt{ednotes}. 
Therefore, ordinary \texttt{.pdf} documentation for 
the remaining \texttt{.sty} files of \lcurl{lineno/}
is missing. 
What you see here is nothing but a somewhat structured listing 
of the additional \texttt{.txt} and \texttt{.sty} files in 
\texttt{PDF}, deriving from the \texttt{verbatim} package and 
its \cs{verbatiminput} command. I hope the high quality 
(scalable) output is worth it. 

\leavevmode\hfill \textit{U.\,L.}

\newpage 
\tableofcontents
 
\section{The \texttt{.txt} files}
\subsection{\texttt{README.txt}}
\verbatiminput{README.txt}
\subsection{\texttt{COPYING.txt}}
\verbatiminput{COPYING.txt}
\subsection{\texttt{CHANGES.txt}}
\verbatiminput{CHANGES.txt}

\section{Tabular and array environments}
\texttt{lineno.sty}'s package options \texttt{edtable}, 
\texttt{longtable}, and \texttt{nolongtablepatch} 
redefine \LaTeX\ tabular and array environments 
such that \texttt{lineno} and \texttt{ednotes} commands 
can be used inside. The code for these options resides 
in separate files at present. We are listing them here. 
\subsection{\texttt{edtable.sty}}
\verbatiminput{edtable.sty}
\subsection{\texttt{ltabptch.sty}}
\verbatiminput{ltabptch.sty}

\section{\cs{linelabel} and notes from \textit{math} mode: 
         \notinaux{\\} \texttt{ednmath0.sty}}
\verbatiminput{ednmath0.sty}

\section{Extended line number references: \texttt{vplref.sty}} 
\texttt{vplref.sty} is input through the \texttt{lineno} 
package option \texttt{addpageno}. This adds page numbers 
to line number references to distant sides---using the 
\texttt{varioref} package from the \LaTeX\ distribution. 
\verbatiminput{vplref.sty} 

\end{document}

end of LNoSuppl.tex 
