% option=test    => boxes
%        dummy   => file
%        final   => file / local run
%
% constante van phantom in definitie ONE: \setchemicaltextwidth 300

%D \module
%D   [       file=ppchtex (m-chemie),
%D        version=1997.03.19,
%D          title=\CONTEXT\ Extra Modules,
%D       subtitle=\PPCHTEX\ (Plain Pictex Context cHemie \TEX),
%D         author=Hans Hagen,
%D           date=\currentdate,
%D      copyright={PRAGMA / Hans Hagen \& Ton Otten},
%D    suggestions={Tobias Burnus, Dirk Kuypers \& Ton Otten}]
%C
%C This module is part of the \CONTEXT\ macro||package and is
%C therefore copyrighted by \PRAGMA. See mreadme.pdf for 
%C details. 

% Deze module ondersteunt het zetten van chemische
% (structuur)formules. Hoewel de macro' zijn afgestemd op
% CONTEXT, zijn ze ook buiten deze zetomgeving te gebruiken.
%
% Dit is, afgezien van updates, de definitieve versie van
% PPCHTEX. Gebruikersgemak, eenvoud, flexibiliteit, en
% snelheid zijn inmiddels redelijk geoptimaliseerd. Dit neemt
% niet weg dat hier en daar nog verbetering mogelijk is. Dit
% zal dan ook nog gebeuren.
%
% Volgende versies zullen tenminste dezelfde functionaliteit
% hebben. We houden ons natuurlijk het recht voor de kwaliteit
% van de output te verbeteren. Daarnaast staan nog op het
% wensenlijstje:
%
% -  optimaliseren in termen van proces-tijd
% -  aanpassen naamgeving van interne macro's
% -  toevoegen van functionaliteit
% -  in \x!-vorm omzetten van GIVES, TB enz.
%
% De mix tussen engels en nederlands lijkt soms verwarrend.
% Meestal zijn verborgen macro's engels en zichtbare macro's
% nederlands. Het gebruik van [ ] en { } sluit aan op andere
% Context-macro's. Hetzelfde geldt voor instellingen en
% \start-\stop-constructies.
%
% De schijnbaar overbodige \bgroup-\egroup constructie
% garandeert aansluiting bij de Context-macro's voor het
% plaatsen van figuren, tabellen en andere floats.
%
% Binnen Context worden de macro's geladen met
% \gebruikextras[chemie]. Daarbij wordt een passende melding
% getoont. Buiten Context genereren we een melding:

\doifundefined{usemodule}
  {\writestatus{loading}{Context Chemical Macro's / 1996.3.1}}

% Er kan gebruik worden gemaakt van PiCTeX of PStricks. Een
% van deze pakketten moet van te voren zijn geladen.
%
% \input prepictex.tex (i.g.v. LaTeX)
% \input pictex.tex
% \input postpictex.tex (i.g.v. LaTeX)
%
% of:
%
% \input multido.tex
% \input pstricks.tex
% \input pst-plot.tex
%
% In \CONTEXT\ kan men de modules m-pictex en m-pstricks
% gebruiken. De eerste module laad of efficiente wijze PiCTeX
% en de tweede module koppelt het PSTRICKS kleurmechanisme
% aan dat van \CONTEXT.
%

% PSTricks: {-\chemicalangle} instead of {*0}, which produces
% faulty ps code when \chemicalangle=0

\startcommands dutch english german

    gotochemical:  naarchemie       gotochemical    zurchemie        
   setupchemical:  stelchemiein     setupchemical   stellechemieein 
   startchemical:  startchemie      startchemical   startchemie     
    stopchemical:  stopchemie       stopchemical    stopchemie      
  definechemical:  definieerchemie  definechemical  definierechemie 
        chemical:  chemie           chemical        chemie          
         toptext:  boventekst       toptext         textueber 
         bottext:  ondertekst       bottext         textunter
         midtext:  middentekst      midtext         textmitte

\stopcommands

\doifundefined{fiverm}  % In the more recent LaTeX versions
  {\font\fiverm=cmr5 }  % \fiverm is no longer (pre)defined.

\doifdefinedelse{beginpicture}           % PiCTeX 
  {\doifdefinedelse{startMPdrawing}
     {\chardef\chemicaldrawingmode=2 }   % MetaPost
     {\chardef\chemicaldrawingmode=0 }}  % raw 
  {\doifdefinedelse{psaxes}
     {\chardef\chemicaldrawingmode=1 }   % PSTricks
     {\chardef\chemicaldrawingmode=3 }}  % unknown

\ifcase\chemicaldrawingmode
  \writestatus{ppchtex}{using PiCTeX}
\or
  \writestatus{ppchtex}{using PSTricks (still experimental)}
  \writestatus{ppchtex}{automatic sizing not (yet) supported}
\or
  \writestatus{ppchtex}{using PiCTeX and MetaPost}
\else
  \writestatus{ppchtex}{load PiCTeX (+pre/post) or PSTricks (+pst_plot) first}
  \bgroup
  \read16 to \exit
  \egroup
  \expandafter\endinput
\fi

% De onderstaande help-informatie (%I) kan worden opgeroepen
% in TeXEdit. De daaropvolgende setup-informatie (%S) kan
% nadat zij is uit deze file is gefilterd met TeXUtil, in
% handleidingen worden gebruikt. In deze file opgenomen
% documentatie (%D en %M) kan worden gebruikt voor een
% technische handleiding. Met %T kunnen templates worden
% gedefinieerd voor TeXEdit.

%I n=Chemie
%I c=\stelchemiein,\chemie
%I
%I Chemische formules kunnen worden gezet met behulp van de
%I onderstaande commando's:
%I
%I buiten $ en $$ :
%I
%I   \chemie[segmenten][symbolen]
%I
%I   \startchemie[instellingen]
%I     \chemie...
%I     \chemie...
%I   \stopchemie
%I
%I en binnen $ en $$:
%I
%I   \chemie{}{}
%I
%I Voor tekst, uitleg en voorbeelde verwijzen we vooralsnog
%I naar de handleiding.
%P
%I Het gedrag van de macro's kan worden ingesteld met:
%I
%I   \stelchemiein[breedte=,hoogte=,links=,rechts=,boven=,
%I     onder=,korps=,schaal=,status=,assenstelsel=,kader=,
%I     variant=,optie=,formaat=,tekstformaat=,resolutie=,
%I     offset=,letter=]
%I
%I Structuren kunnen worden voorgedefinieerd met het commando
%I
%I   \definieerchemie[naam]{\chemie...}

%S \startsetup
%S   \command
%S     [\!stelchemiein]
%S   \type
%S     [\c!vars!]
%S   \variable
%S     [\c!breedte]
%S     [\c!number!,\v!passend]
%S     [0]
%S   \variable
%S     [\c!hoogte]
%S     [\c!number!,\v!passend]
%S     [0]
%S   \variable
%S     [\c!links]
%S     [\c!number!]
%S     [0]
%S   \variable
%S     [\c!rechts]
%S     [\c!number!]
%S     [0]
%S   \variable
%S     [\c!boven]
%S     [\c!number!]
%S     [0]
%S   \variable
%S     [\c!onder]
%S     [\c!number!]
%S     [0]
%S   \variable
%S     [\c!resolutie]
%S     [\c!number!]
%S     [\outputresolution]
%S   \variable
%S     [\c!korps]
%S     [10pt,11pt,12pt]
%S     [\bodyfontsize]
%S   \variable
%S     [\c!schaal]
%S     [\v!klein,\v!middel,\v!groot]
%S     [\v!middel]
%S   \variable
%S     [\c!formaat]
%S     [\v!klein,\v!middel,\v!groot]
%S     [\v!groot]
%S   \variable
%S     [\c!tekstformaat]
%S     [\v!klein,\v!middel,\v!groot]
%S     [\v!groot]
%S   \variable
%S     [\c!status]
%S     [\v!start,\v!stop]
%S     [\v!start]
%S   \variable
%S     [\c!kader]
%S     [\v!aan,\v!uit]
%S     [\v!uit]
%S   \variable
%S     [\c!assenstelsel]
%S     [\v!aan,\v!uit]
%S     [\v!uit]
%S   \variable
%S     [\c!optie]
%S     [\v!test]
%S     []
%S   \variable
%S     [\c!variant]
%S     [1,2]
%S     [1]
%S   \variable
%S     [\c!offset]
%S     [HIGH,LOW]
%S     [LOW]
%S   \variable
%S     [\c!letter]
%S     [\c!command!]
%S     [\rm]
%S \stopsetup

%S \startsetup
%S   \command
%S     [\v!startchemie]
%S   \type
%S     [\c!vars!\c!stp!]
%S   \inheritvariable
%S     [\v!stelchemiein]
%S     []
%S \stopsetup

%S \startsetup
%S   \command
%S     [\v!chemie]
%S   \type
%S     [\c!vals!\c!vals!]
%S   \value
%S     [\c!list!]
%S     []
%S   \value
%S     [\c!list!]
%S     []
%S \stopsetup

%S \startsetup
%S   \command
%S     [definieerchemie]
%S   \type
%S     [\c!val!\c!arg!]
%S   \value
%S     [\c!text!]
%S     []
%S \stopsetup

\unprotect

% Om te voorkomen dat sub- en superscripts botsen passen we
% wat fontdimen's aan (Knuth, The TeXBook, p179). Helaas
% kunnen deze instellingen niet lokaal worden gehouden door
% groeperen, vandaar dat een en ander moet worden geset �n
% gereset.
%
% Er dient een relatie te worden gelegd met de afmetingen
% van de letters. In een eerdere versie werden daartoe de
% \fontdimen's opgehoogd. Omdat dit problemen gaf bij
% scaled fonts, is bij nader inzien gekozen voor de
% onderstaande oplossing, waarbij de nieuwe waarden worden
% afgeleid van de x-height (\fontdimen5). De factor 0.70
% is min of meer experimenteel vastgesteld. Soms worden de
% regels iets verder uit elkaar gezet. Jammer. Italic fonts
% hebben grotere cijfers en vallen min of meer uit de boot.

\newif\ifloweredsubscripts

% Due to some upward incompatibality of LaTeX to LaTeX2.09
% and/or LaTeX2e we had to force \@@chemicalletter. Otherwise
% some weird \nullfont error comes up.

\def\beginlatexmathmodehack%
  {\ifmmode
     \let\endlatexmathmodehack=\relax
   \else
     \def\endlatexmathmodehack{$}$\@@chemicalletter
   \fi}

\def\setsubscripts%
  {\beginlatexmathmodehack
   \def\dosetsubscript##1##2##3%
     {\dimen0=##3\fontdimen5##2%
      \setxvalue{@@\string##1\string##2}{\the##1##2\relax}%
      ##1##2=\dimen0\relax}%
   \def\dodosetsubscript##1##2%
     {\dosetsubscript{##1}{\textfont2}{##2}%
      \dosetsubscript{##1}{\scriptfont2}{##2}%
      \dosetsubscript{##1}{\scriptscriptfont2}{##2}}%
   %\dodosetsubscript{\fontdimen14}{?}%
   \dodosetsubscript{\fontdimen16}{.7}%
   \dodosetsubscript{\fontdimen17}{.7}%
   \global\loweredsubscriptstrue
   \endlatexmathmodehack}

\def\resetsubscripts%
  {\ifloweredsubscripts
     \beginlatexmathmodehack
     \def\doresetsubscript##1##2%
       {\dimen0=\getvalue{@@\string##1\string##2}\relax
        ##1##2=\dimen0}%
     \def\dodoresetsubscript##1%
       {\doresetsubscript{##1}{\textfont2}%
        \doresetsubscript{##1}{\scriptfont2}%
        \doresetsubscript{##1}{\scriptscriptfont2}}%
     %\dodoresetsubscript{\fontdimen14}%
     \dodoresetsubscript{\fontdimen16}%
     \dodoresetsubscript{\fontdimen17}%
     \global\loweredsubscriptsfalse
     \endlatexmathmodehack
   \fi}

\def\doresetsubscripts%
  {\resetsubscripts}

\def\sethighsubscripts%
  {\resetsubscripts
   \let\dosetsubscripts=\relax}

\def\setlowsubscripts%
  {\def\dosetsubscripts{\setsubscripts}}

\setlowsubscripts

\newcount\horchemical % t.z.t. \newcounter
\newcount\verchemical % t.z.t. \newcounter
\newcount\txtchemical % t.z.t. \newcounter
\newcount\levchemical % t.z.t. \newcounter

\newif\ifinchemical            \inchemicalfalse
\newif\iffixedchemical         \fixedchemicalfalse

\newbox\chemicalsymbols

% Eigenlijk moeten de constanten en variabelen in cont-nl.tex
% staan. Dit pakket is echter relatief onafhankelijk van CONTEXT.

\definesystemvariable  {chemical}

\definesystemconstant  {chemical}

\definesystemconstant  {translate}
\definesystemconstant  {distance}
\definesystemconstant  {mirror}
\definesystemconstant  {rotate}
\definesystemconstant  {substitute}
\definesystemconstant  {angle}

\definesystemconstant  {executechemical}
\definesystemconstant  {chemicaltextelement}
\definesystemconstant  {chemicallinesegment}
\definesystemconstant  {chemicalcircsegment}

\def\chemicalspace     {\quad}

% begin van experiment:
%
% De onderstaande twee macro's kunnen worden gebruikt voor
% bijvoorbeeld een interactiemechanisme.
%
% \localgotochemical   {verwijzing} {tekst}
% \localthisischemical {verwijzing}

\def\dowithchemical%
  {}

\doifdefinedelse{@@iastatus}
  {\def\localgotochemical#1#2{\naarbox{#2}[#1]}%
   \def\localthisischemical#1{\pagereference[#1]}}
  {\def\localgotochemical#1{}%
   \def\localthisischemical#1{}}

% eind van experiment

\def\setchemicalmaximum #1
  {\def\maxchemical{#1}}

\def\doifchemicalnumber#1#2#3%
  {\doifnumberelse{#1}
     {\ifnum#1>\maxchemical
        \writestatus{ppchtex}{number #1 is skipped}%
      \else
        #3%
      \fi}
     {\unknownchemical{#2}}}%

\newif\ifsmallchemicaltext

\def\setupchemicalformat[#1]%
  {\processaction
     [\getvalue{#1\c!formaat}]
     [  \v!klein=>\def\@@localchemicalformat%
                    {\scriptscriptstyle},
       \v!middel=>\def\@@localchemicalformat%
                    {\ifsmallchemicaltext\scriptscriptstyle\else\scriptstyle\fi},
        \v!groot=>\def\@@localchemicalformat
                    {\ifsmallchemicaltext\scriptstyle\else\textstyle\fi},
      \s!unknown=>\def\@@localchemicalformat%
                    {\getvalue{#1\c!formaat}}]%
   \processaction
     [\getvalue{#1\c!tekstformaat}]
     [  \v!klein=>\def\@@localchemicalstyle%
                    {\scriptscriptstyle},
       \v!middel=>\def\@@localchemicalstyle%
                    {\ifsmallchemicaltext\scriptscriptstyle\else\scriptstyle\fi},
        \v!groot=>\def\@@localchemicalstyle%
                    {\ifsmallchemicaltext\scriptstyle\else\textstyle\fi},
      \s!unknown=>\def\@@localchemicalstyle%
                    {\getvalue{#1\c!tekstformaat}}]%
   \processaction
     [\getvalue{#1\c!schaal}]
     [  \v!klein=>\def\@@localchemicalscale{500},
       \v!middel=>\def\@@localchemicalscale{625},
        \v!groot=>\def\@@localchemicalscale{750},
      \s!unknown=>\def\@@localchemicalscale{\getvalue{#1\c!schaal}}]}

\def\@@currentchemicalformat%
  {\ifinchemical
     \@@localchemicalformat
   \else
     \@@localchemicalstyle
   \fi}

\def\dosetupchemical[#1]%
  {\getparameters[\??chemical\s!chemical][#1]%
   \doifelse{\@@chemicalchemicaloffset}{LOW}
     {\setlowsubscripts}
     {\sethighsubscripts}%
   \setupchemicalformat[\??chemical\s!chemical]}

\def\setupchemical%
  {\dosingleargument\dosetupchemical}

\def\@@chemicalletter%          %  $inner-style$
  {\@@chemicalchemicalletter}   % $$outer-style$$

\def\@@writechemicalstatus#1#2%
  {}

\def\@@beginchemicallocalpicture%
  {\ifcase\chemicaldrawingmode
     \beginpicture
   \or
     \pspicture(0,0)(0,0) % is this permitted ?
   \or
     \pushMPdrawing
     \startMPdrawing
       %prologues := 1 ; 
       input mp-tool ; 
       u := 10*\@@chemicalunit;
       bboxmargin := 0pt ; 
       pickup pencircle scaled 2u ; % ??? 
     \stopMPdrawing
     \beginpicture
   \fi}

\def\@@endchemicallocalpicture#1#2%
  {\ifcase\chemicaldrawingmode
     \endpicture
   \or
     \endpspicture
   \or
     \resetchemicalcoordinates
     \setbox2=\hbox{\MPshiftdrawingtrue\getMPdrawing}%
     \wd2=\!!zeropoint\ht2=\!!zeropoint\dp2=\!!zeropoint
     \put {\box2} at 0 0 
     \endpicture 
     \popMPdrawing
   \fi}

\def\@@beginchemicalpicture#1#2#3#4%
  {\ifnum\chemicaldrawingmode=1
     \pspicture(#1,#3)(#2,#4)%
     \def\account##1##2{}%
     \psaxes[axesstyle=none,labels=none,ticks=none](#1,#3)(#2,#4)%
   \else
     \beginpicture
     \setplotarea
       x from {#1} to {#2},
       y from {#3} to {#4}
     \iffixedchemical
       \accountingon
       \def\account##1##2%
         {\put {} at {##1} {##2} }%
     \else
       \accountingoff
       \def\account##1##2{}%
     \fi
   \fi
   \ignorespaces}

\def\@@endchemicalpicture%
  {\ifcase\chemicaldrawingmode
     \put {\box\chemicalsymbols} at 0 0 %  elders
     \endpicture
   \or
     \rput(0,0){\box\chemicalsymbols}%
     \endpspicture
   \or
     \put {\box\chemicalsymbols} at 0 0 %  elders
     \ifMPdrawingdone
       \resetchemicalcoordinates
       \setbox2=\hbox{\MPshiftdrawingtrue\getMPdrawing}%
       \wd2=\!!zeropoint\ht2=\!!zeropoint\dp2=\!!zeropoint
       \put {\box2} at 0 0 %
     \fi
     \endpicture
   \fi}

\def\@@setchemicalcoordinatesystem#1%
  {\edef\@@chemicalunit{#1}%
   \ifcase\chemicaldrawingmode
     \setcoordinatesystem units <\@@chemicalunit,\@@chemicalunit> %
   \or
     \psset{unit=\@@chemicalunit}%
   \or
     \setcoordinatesystem units <\@@chemicalunit,\@@chemicalunit> %
     \startMPdrawing
       input mp-tool ; 
      %prologues := 1 ; 
       u := 10*#1;
       bboxmargin := 0pt ; 
       pickup pencircle scaled 2u ; % ??? 
     \stopMPdrawing
   \fi}

\def\@@setchemicalaxis#1#2#3#4%
  {\ifcase\chemicaldrawingmode
     \axis
       bottom shiftedto y=0
       ticks from {#1} to {#2} by 500 /
     \axis
       left shiftedto x=0
       ticks from {#3} to {#4} by 500 / %
   \or
     \psaxes[labels=none,Dx=500,Dy=500](0,0)(#1,#3)(#2,#4)%
   \or
     \global\MPdrawingdonetrue
     \startMPdrawing
        x1 := \MPdivten[#1]u ; x2 := \MPdivten[#2]u;
        y1 := \MPdivten[#3]u ; y2 := \MPdivten[#4]u;
        draw z1--(x2,y1)--z2--(x1,y2)--cycle ;
        d := 50u ; dd := 10u ;
        draw (x1,0)--(x2,0) ;
        draw (0,y1)--(0,y2) ;
        for i=d step -d until x1: draw (i,dd)--(i,-dd) ; endfor ;  
        for i=d step  d until x2: draw (i,dd)--(i,-dd) ; endfor ;  
        for i=d step -d until y1: draw (-dd,i)--(dd,i) ; endfor ;  
        for i=d step  d until y2: draw (-dd,i)--(dd,i) ; endfor ;  
     \stopMPdrawing
   \fi}

\def\@@setsecondchemicalplotsymbol%
  {\ifcase\chemicaldrawingmode
     \!!widtha=50.8mm
     \divide\!!widtha by \@@chemicalresolutie\relax
     \plotsymbolspacing=\!!widtha
     \setplotsymbol({\vrule\!!height\!!widtha\!!width\!!widtha})%
   \fi}

% Something for Dirk: 

\newcount \currentchemical

%\newif    \ifskipchemical
%
\def\setchemicaldimensions#1#2#3%
  {\bgroup
   \global\advance\currentchemical by 1
   \dimen0=#1\relax
   \dimen2=#2\relax
   \dimen4=#3\relax
   \setxvalue{chemical::\the\currentchemical}%
     {\noexpand\docommando{\the\dimen0}{\the\dimen2}{\the\dimen4}}%
   \egroup}
%
%\def\getchemicaldimensions#1#2#3%
%  {\global\advance\currentchemical by 1
%   \def\docommando##1##2##3%
%     {#1=##1\relax#2=##2\relax#3=##3\relax}%
%   \doifdefinedelse{chemical::\the\currentchemical}
%     {\getvalue{chemical::\the\currentchemical}}
%     {\docommando{6cm}{4cm}{0cm}}}
%
%\def\savechemicaldimensions%
%  {\bgroup
%   \writestatus{ppchtex}{saving dimensions in ppchtex.dim}%
%   \def\docommando##1##2##3%
%     {\immediate\write\scratchwrite 
%        {\noexpand\setchemicaldimensions{##1}{##2}{##3}}}%
%   \immediate\openout\scratchwrite=ppchtex.dim
%   \scratchcounter=0
%   \loop
%     \ifnum\scratchcounter<\currentchemical
%       \advance\scratchcounter by 1
%       \getvalue{chemical::\the\scratchcounter}%
%   \repeat
%   \immediate\closeout\scratchwrite
%   \egroup}
%
%\def\loadchemicaldimensions% oh, how nice it would be to use
%  {\bgroup                 % one of the context read commands 
%   \global\currentchemical=0
%   \immediate\openin\scratchread=./ppchtex.dim
%   \ifeof\scratchread 
%     \immediate\closein\scratchread
%     \global\skipchemicalfalse
%   \else
%     \immediate\closein\scratchread
%     \input ./ppchtex.dim\relax
%     \ifnum\currentchemical>0
%       \writestatus{ppchtex}{loading dimensions from ppchtex.dim}%
%       \global\skipchemicaltrue
%     \else
%       \global\skipchemicalfalse
%     \fi
%     \global\currentchemical=0
%     \global\let\savechemicaldimensions=\relax
%   \fi
%   \egroup
%   \global\let\loadchemicaldimensions=\relax}

\ifx\normalchemicalframe\undefined
  \let\normalchemicalframe\hbox % hook for educational purposes
\fi

\unexpanded\def\complexstartchemical[#1]%
  {\copyparameters
     [\??chemical][\??chemical\s!chemical]
     [\c!breedte,\c!hoogte,\c!links,\c!rechts,\c!boven,\c!onder,
      \c!korps,\c!formaat,\c!schaal,\c!status,\c!kader,\c!assenstelsel,
      \c!plaats,\c!optie,\c!variant,\c!resolutie,\c!offset,\c!letter]%
   \getparameters
     [\??chemical]
     [#1]%
   %
   \setupchemicalformat[\??chemical]%
   %
   \ifnum\chemicaldrawingmode=2
     \resetMPdrawing
   \fi
   % 
   \doif{\@@chemicalvariant}{2}
     {\@@setsecondchemicalplotsymbol}%
   %
   \!!counta=250000
   \divide\!!counta by \@@localchemicalscale
   \!!widtha=\@@chemicalkorps
   \divide\!!widtha by \!!counta
   \@@setchemicalcoordinatesystem{\the\!!widtha}%
   %
   % \!!counta = -x     \!!countc = -y
   % \!!countb = +x     \!!countd = +y
   %
   \def\calculateaxis##1##2##3##4##5%
     {##1=##3\relax
      ##2=##4\relax
      \ifnum##5=0
        \ifnum##3=0
          \ifnum##4=0
            ##1=2000
            ##2=2000
           \fi
        \fi
      \else
        \ifnum##3=0
          \ifnum##4=0
            ##1=##5\relax
            \divide##1 by 2
            ##2=##1\relax
          \else
            ##1=##5\relax
            \advance##1 by -##2\relax
          \fi
        \else
          \ifnum##4=0
            ##2=##5\relax
            \advance##2 by -##1\relax
          \fi
        \fi
      \fi}%
   \fixedchemicalfalse
   \doif{\@@chemicalbreedte}{\v!passend}
     {\edef\@@chemicalbreedte%
        {\ifnum\chemicaldrawingmode=1 2000 \else 1 \fi}%
      \fixedchemicaltrue}%
   \doif{\@@chemicalhoogte}{\v!passend}
     {\edef\@@chemicalhoogte%
        {\ifnum\chemicaldrawingmode=1 2000 \else 1 \fi}%
      \fixedchemicaltrue}%
   \doifelse{\@@chemicalplaats}{\v!intekst}
     {\!!counta=0 \!!countb=0
      \!!counta=0 \!!countd=0 }
     {\calculateaxis
        {\!!counta}{\!!countb}
        {\@@chemicallinks}{\@@chemicalrechts}{\@@chemicalbreedte}%
      \calculateaxis
        {\!!countc}{\!!countd}
        {\@@chemicalonder}{\@@chemicalboven}{\@@chemicalhoogte}}%
   %
   \edef\@@chemheight   {\the\!!countc}%
   \edef\@@chemdepth    {\the\!!countd}%
   \edef\@@chemicalboven{\the\!!countc}%
   \edef\@@chemicalonder{\the\!!countd}%
   %
   \doifinsetelse{\v!aan}{\@@chemicalkader,\@@chemicalassenstelsel}
     {\def\@@chemicalborder{\chemicalframe}}
     {\def\@@chemicalborder{\normalchemicalframe}}%
   %
   \setbox0=\hbox\bgroup    % this was a \vbox which took \hsize
   %
   \@@beginchemicalpicture
     {-\the\!!counta}{\the\!!countb}
     {-\the\!!countc}{\the\!!countd}%
   \doif{\@@chemicalstatus}{\v!start}
     {\doif{\@@chemicalassenstelsel}{\v!aan}
        {\@@setchemicalaxis
           {-\the\!!counta}{\the\!!countb}
           {-\the\!!countc}{\the\!!countd}}}%
   \doifelse{\@@chemicaloptie}{\v!test}
     {\def\@@writechemicalstatus##1##2% 
        {\convertargument##2\to\ascii
         \writestatus{##1}{\ascii}}}
     {\def\@@writechemicalstatus##1##2{}}%
   \ignorespaces}

\def\dostartchemical%
  {\catcode`\^=\@@superscript% t.b.v. \enableduplication
   \catcode`\_=\@@subscript  % t.b.v. de zekerheid
   \begingroup
   \inchemicaltrue
   \def\toptext##1{\gdef\thetoptext{##1}\ignorespaces}\toptext{}%
   \def\bottext##1{\gdef\thebottext{##1}\ignorespaces}\bottext{}%
   \def\midtext##1{\gdef\themidtext{##1}\ignorespaces}\midtext{}%
   \def\@@chemicalpostponed{}%
   \complexorsimpleempty\startchemical}

\def\startchemical%
  {\bgroup % t.b.v. ungrouped floats
%   \loadchemicaldimensions
%   \ifskipchemical
%     \def\dostartchemical%
%       {\def\dummy[####1]{}\dosingleempty\dummy}%
%     \def\chemical%
%       {\def\dummy[####1][####2][####3]{}\dotripleempty\dummy}%
%     \def\toptext##1{}%
%     \def\midtext##1{}%
%     \def\bottext##1{}%
%   \fi
   \dostartchemical}

\def\stopchemical%
  {%\ifskipchemical
   %  \getchemicaldimensions{\dimen0}{\dimen2}{\dimen4}%
   %  \dimen8=\dimen2\advance\dimen8 by \dimen4
   %  \setbox0=\vbox to \dimen8
   %    {\vss\hbox to \dimen0{\hss\the\currentchemical\hss}\vss}%
   %  \wd0=\dimen0\ht0=\dimen2\dp0=\dimen4
   %  \chemicalframe{\box0}%
   %\else
     \checkchemicalpicture
     \@@endchemicalpicture
     \egroup
     \ifnum\chemicaldrawingmode=1
       \dimen0=\@@chemicalunit
       \setbox0=\hbox{\lower\@@chemdepth\dimen0\box0}%
       \ht0=\@@chemheight\dimen0
       \dp0=\@@chemdepth\dimen0
     \fi
     \dimen0=\ht0
     \advance\dimen0 by \dp0
     \inchemicalfalse % enables \chemie{} in text 
     \setbox4=\alignedchemical\themidtext
     \setbox6=\alignedchemical\thetoptext
     \setbox8=\alignedchemical\thebottext
     \setbox4=\hbox to \wd0
       {\strut\hss$\vcenter{\box4}$\hss}%
     \setbox2=\vbox to \dimen0
       {\hbox to \wd0{\strut\hss\box6\hss}
        \vfill
        \hbox to \wd0{\strut\hss\box8\hss}
        \vss}% disables the depth
     \wd0=0pt  \wd4=0pt
     \ht2=\ht0 \dp2=\dp0
     \ht4=\ht0 \dp4=\dp0 
    %\setchemicaldimensions{\wd2}{\ht2}{\dp2}%  
     \@@chemicalborder{\box0\box4\box2}% text on top of chemicals
     \endgroup
   %\fi
   \ignorespaces
   \egroup}                  % t.b.v. ungrouped floats

\def\alignedchemical#1%
  {\vtop
     {\def\par{\egroup\hbox\bgroup\strut}%
      \let\\=\par
      \let\endgraf=\par
      \hbox\bgroup\strut#1\egroup}}

% \setchemicalcoordinates{#1}{#2}
%
% #1: verplaatsing in x-richting
% #2: verplaatsing in y-richting

\newif\ifchemicaldirection

\def\checkchemicaldirection#1#2%
  {\ifchemicaldirection
     % \ifnum#1>0 \advance\horchemical by +\chemicaldirection \fi
     % \ifnum#1<0 \advance\horchemical by -\chemicaldirection \fi
     % \ifnum#2>0 \advance\verchemical by +\chemicaldirection \fi
     % \ifnum#2<0 \advance\verchemical by -\chemicaldirection \fi
     \ifnum#1>0 \advance\horchemical by -\chemicaldirection \fi
     \ifnum#1<0 \advance\horchemical by +\chemicaldirection \fi
     \ifnum#2>0 \advance\verchemical by -\chemicaldirection \fi
     \ifnum#2<0 \advance\verchemical by +\chemicaldirection \fi
     \chemicaldirectionfalse
   \fi}

\def\processchemicaldirection%
  {\chemicaldirectiontrue\processchemicaltranslate}

\def\setchemicalcoordinates#1#2%
  {\advance\horchemical by #1\relax
   \advance\verchemical by #2\relax
   \checkchemicaldirection{#1}{#2}%
   \!!counta=-\horchemical\edef\chemicalxoffset{\the\!!counta}%
   \!!countb=-\verchemical\edef\chemicalyoffset{\the\!!countb}%
   \ifnum\chemicaldrawingmode=1 
     % njet 
   \else
     \setcoordinatesystem point at {\the\horchemical} {\the\verchemical}
   \fi}

\def\resetchemicalcoordinates%
  {\horchemical=0
   \verchemical=0
   \edef\chemicalxoffset{0}%
   \edef\chemicalyoffset{0}%
   \ifnum\chemicaldrawingmode=1 
     % njet 
   \else
     \setcoordinatesystem point at 0 0
   \fi}

\def\restorechemicalcoordinates%
  {%\writestatus{ppchtex}{restoring \the\horchemical,\the\verchemical}%
   \edef\chemicalxoffset{\the\horchemical}%
   \edef\chemicalyoffset{\the\verchemical}%
   \ifnum\chemicaldrawingmode=1 
     % njet 
   \else
     \setcoordinatesystem point at {\the\horchemical} {\the\verchemical}
   \fi}

\def\setchemicaltranslate #1 #2 #3
  {\setvalue{\s!translate#1}{\setchemicalcoordinates{#2}{#3}}}

\def\processchemicaltranslate#1%
  {\def\doprocess[##1##2]%
     {\doifchemicalnumber{##1}{MOV#1}
        {\ifnum##1=0
           \def\chemicaloffset{0}% incompatible change
           \resetchemicalcoordinates
         \else
           \getvalue{\s!translate##1}%
           \dochemicaloffset{##1}%
           \def\chemicaloffset{0}%
         \fi}}%
   \doprocess[#1]}

\def\setchemicaldistance #1
  {\setvalue{\s!distance1}{\setchemicalcoordinates{-#1}{  0}}%
   \setvalue{\s!distance2}{\setchemicalcoordinates{  0}{ #1}}%
   \setvalue{\s!distance3}{\setchemicalcoordinates{ #1}{  0}}%
   \setvalue{\s!distance4}{\setchemicalcoordinates{  0}{-#1}}}

\def\setchemicaldirection #1
  {\def\chemicaldirection{#1}}

\def\processchemicaldistance#1%
  {\def\doprocess[##1##2]%
     {\doifchemicalnumber{##1}{ADJ#1}
        {\ifnum##1=0
           \resetchemicalcoordinates
         \else
           \def\@@chemicalpostponed{\getvalue{\s!distance##1}}%
           \@@chemicalpostponed
         \fi}}%
   \doprocess[#1]}

\def\setchemicalsubstitute #1
  {\setvalue{\s!substitute1}{\setchemicalcoordinates{-#1}{  0}}%
   \setvalue{\s!substitute2}{\setchemicalcoordinates{  0}{ #1}}%
   \setvalue{\s!substitute3}{\setchemicalcoordinates{ #1}{  0}}%
   \setvalue{\s!substitute4}{\setchemicalcoordinates{  0}{-#1}}}

\def\processchemicalsubstitute#1%
  {\def\doprocess[##1##2]%
     {\doifchemicalnumber{##1}{SUB#1}
        {\ifnum##1=0
           \resetchemicalcoordinates
         \else
           \def\@@chemicalpostponed{\getvalue{\s!substitute##1}}%
           \@@chemicalpostponed
         \fi}}%
   \doprocess[#1]}

% Het is mogelijk een offset of move meerdere malen uit te
% voeren, door een nummer voor het commando te plaatsen.

\def\chemicalrepeat       {1}

\def\redoprocesschemical[#1#2]%
  {\doifinstringelse{#1}{0123456789.}
     {\edef\chemicalrepeat{\chemicalrepeat#1}%
      \redoprocesschemical[#2]}
     {\processchemical[#1#2]%
      \def\chemicalrepeat{1}}}

\def\doprocesschemical[#1#2]#3%
  {\doifinstringelse{#1}{0123456789.}
     {\def\chemicalrepeat{#1}%
      \redoprocesschemical[#2]}
     {#3}}

% \dochemicaloffset{#1}
%
% #1: binding

\def\chemicaloffset{0}

\def\processchemicaloffset#1%
  {\dimen0=62500 sp  % real calc on cardinals
   \dimen0=\chemicalrepeat\dimen0
   \divide\dimen0 by \@@localchemicalscale
   \!!counta=\dimen0
   \def\doprocess[##1##2]%
     {\doifinstringelse{##1}{128}
        {\edef\chemicaloffset{\the\!!counta}}
        {\doifinstringelse{##1}{456}
           {\edef\chemicaloffset{-\the\!!counta}}
           {\doifelse{##1}{0}
              {\edef\chemicaloffset{0}}
              {\unknownchemical{OFF#1}}}}}%
   \doprocess[#1]}

\def\dochemicaloffset#1%
  {\ifnum\chemicaloffset=0
     \def\undochemicaloffset{}%
   \else
     \setchemicalcoordinates{-\chemicaloffset}{0}%
     \def\undochemicaloffset%
       {\setchemicalcoordinates{\chemicaloffset}{0}%
        \def\undochemicaloffset{}}%
   \fi}

\def\processchemicalphantom#1#2% 
  {\setbox0=\hbox
     {\def\splitoff##1????{##1}% 
      $\@@chemicalletter{\@@localchemicalformat\splitoff#2}$}%
   \dimen0=.25\wd0
   \divide\dimen0 by \@@localchemicalscale
   \!!counta=\dimen0
   \doifinstringelse{#1}{128}
     {\edef\chemicaloffset{\the\!!counta}}
     {\doifinstringelse{#1}{456}
        {\edef\chemicaloffset{-\the\!!counta}}
        {\doifelse{#1}{0}
           {\edef\chemicaloffset{0}}
           {\unknownchemical{OF#1:#2}}}}}

% \dosetchemicalrotation{#1}{#2}
%
% #1: cos(phi)
% #2: sin(phi)

\def\chemicalrotation   {1}
\def\chemicalangle      {0}
\def\chemicalxoffset    {0}
\def\chemicalyoffset    {0}

\def\setchemicalmirror#1%
  {\setvalue{\s!mirror#1}{*}}

\def\resetchemicalmirror#1%
  {\resetvalue{\s!mirror#1}}

\def\togglechemicalmirror#1%
  {\doifelse{\getvalue{\s!mirror#1}}{*}
     {\resetchemicalmirror{#1}}
     {\setchemicalmirror{#1}}}

\def\setchemicalrotation #1 #2 #3 #4 #5 #6 #7 #8 #9
  {\setvalue{\s!rotate1.#1}{\dosetchemicalrotation{#2}{#3}}%
   \setvalue{\s!rotate2.#1}{\dosetchemicalrotation{#4}{#5}}%
   \setvalue{\s!rotate3.#1}{\dosetchemicalrotation{#6}{#7}}%
   \setvalue{\s!rotate4.#1}{\dosetchemicalrotation{#8}{#9}}}

\def\setchemicalangle #1 #2 #3 #4 #5
  {\setvalue{\s!angle1.#1}{\dosetchemicalangle{#2}}%
   \setvalue{\s!angle2.#1}{\dosetchemicalangle{#3}}%
   \setvalue{\s!angle3.#1}{\dosetchemicalangle{#4}}%
   \setvalue{\s!angle4.#1}{\dosetchemicalangle{#5}}}

\def\chemicalrotate[#1]%
  {\doifdefinedelse{\s!mirror#1}
     {\getvalue{\s!rotate\chemicalrotation.#1\getvalue{\s!mirror#1}}%
      \getvalue{\s!angle\chemicalrotation.#1\getvalue{\s!mirror#1}}}
     {\getvalue{\s!rotate\chemicalrotation.#1}%
      \getvalue{\s!angle\chemicalrotation.#1}}}

\def\dosetchemicalangle#1%   zwak zie onder
  {\def\chemicalangle{#1}}

\def\dosetchemicalrotation#1#2%
  {\ifnum\chemicaldrawingmode=1 
     % njet 
   \else
     \startrotation by {#1} {#2} %% \stoprotation (t.b.v. testen)
   \fi}

\def\doresetchemicalrotation%
  {\ifnum\chemicaldrawingmode=1 
     % njet 
   \else
     \stoprotation
   \fi}

\def\processchemicalrotation#1%
  {\def\doprocess[##1##2]%
     {\doifnumberelse{##1}
        {\def\chemicalrotation{##1}}
        {\unknownchemical{ROT#1}}}%
   \doprocess[#1]}

% \filtertextelement[#1][#2][#3][#4]
%
% #1: volgnummer
% #2: offset in uitlijningen
% #3: lijst met uitlijningen -> \chemicalloca
% #4: lijst met teksten      -> \chemicaltext

\def\setchemicallocation#1%
  {\doifelse{#1}{}
     {\edef\chemicalloca{c}}
     {\edef\chemicalloca{#1}}}

\newif\iffixedchemicaltext

\def\filterchemicaltextelement[#1][#2][#3][#4]%
  {\ifchemicaltextconstant
     \def\chemicaltext{#4}%
     \setchemicallocation{}%
   \else
     \ifnum#1=0\relax
       \setchemicallocation{}%
     \else
       \iffixedchemicaltext
         \!!counta=#2
       \else
         \!!counta=\chemicalrotation
         \advance\!!counta by -1
         \multiply\!!counta by #2
         \advance\!!counta by #1
       \fi
       \getfromcommalist[#3][\the\!!counta]%
       \setchemicallocation\commalistelement
     \fi
     \ifchemicalpicture
       \let\chemicaltext=\relax
     \else
       \advance\txtchemical by 1
       \getfromcommalist[#4][\txtchemical]%
       \let\chemicaltext=\commalistelement
     \fi
   \fi
   \fixedchemicaltextfalse}

% \putchemicaltext{#1}{#2}
%
% #1 : x-coordinaat
% #2 : y-coordinaat
%
% \chemicaltext en \chemicalloca worden met \gettextelement
% opgehaald uit de tweede set bij \chemie
%
% Ten behoeve van testdoeleinden wordt gebruik gemaakt van
% \chemicalframe in plaats van het meer sjieke, maar tevens
% meer trage \framed.

\ifx\ruledhbox\undefined
  \def\chemicalframe#1%
     {\hbox
        {\vrule\hskip-.4pt
         \vbox{\hrule\vskip-.4pt\hbox{#1}\vskip-.4pt\hrule}%
         \hskip-.4pt\vrule}}
\else
  \def\chemicalframe#1%
    {\ruledhbox{#1}}
\fi

\def\doputchemicaltext#1 [#2] at #3 #4 %
  {\ifnum\chemicaldrawingmode=1
     \rput
       {\chemicalangle}(\chemicalxoffset,\chemicalyoffset)%
       {\expanded{\rput[#2]{-\chemicalangle}(#3,#4){#1}}}%
   \else
     \put {#1} [#2] at {#3} {#4} %
   \fi}

\def\dodoifsinglelocation#1#2\\#3%
  {\ifx#2\relax#3\fi}

\def\doifsinglelocationelse#1%
  {\expandafter\dodoifsinglelocationelse#1\relax\\}

\def\putchemicaltext#1#2%
  {\enablechemicalspecials
   \ifchemicalpicture
     \setchemicalpicture{#1}{#2}%
   \else
     \doifelse{\@@chemicaloptie}{\v!test}
       {\def\@@chemicalframe{\chemicalframe}}
       {\def\@@chemicalframe{}}%
     \dosetsubscripts
     \setbox2=\hbox{$\@@chemicalletter{\@@localchemicalformat \chemicaltext}$}%
     \setbox4=\hbox{$\@@chemicalletter{\@@localchemicalformat C_2^2}$}%
     \setbox6=\hbox{$\@@chemicalletter{\@@localchemicalformat O}$}% or C
     \doresetsubscripts
     \doifnot{\@@chemicalplaats}{\v!intekst}
       {\ht2=\ht4
        \dp2=\dp4}%
     \setbox2=\hbox{\@@chemicalframe{\box2}}%
     \ifdim\wd2>\wd6
       \doifelse{#1}{0}
         {\doifnot{#2}{0}{\wd2=\wd6}}
         {%\doifsinglelocation\chemicalloca
            {\doifinset{\chemicalloca}{t,b}{\wd2=\wd6}}}% common ?
     \fi
     \expanded
       {\doputchemicaltext
          {\noexpand\dowithchemical{\copy2}}     % per se \copy2 i.p.v. \box2
          [\chemicalloca] at {#1} {#2} }
     \nomoreaccounting
   \fi
   \disablechemicalspecials}

\def\setchemicaltextelement #1 #2 #3
  {\setvalue{\s!chemicaltextelement#1}{\putchemicaltext{#2}{#3}}}

\def\getchemicalfixedtextelement%
  {\fixedchemicaltexttrue
   \getchemicaltextelement}

\def\getchemicaltextelement[#1][#2][#3][#4][#5]%
  {\filterchemicaltextelement[#2][#3][#4][#5]%
   \doifelse{#2}{0}
     {\dochemicaloffset{#2}%     % incompatible change
      \putchemicaltext{0}{0}%
      \undochemicaloffset}       % incompatible change
     {\chemicalrotate[#2]%
      \dochemicaloffset{#2}%
      \def\chemicaltextelementnumber{#2}%
      \getvalue{\s!chemicaltextelement#1}%
      \getvalue{\s!chemicaltextelement#11}%
      \getvalue{\s!chemicaltextelement#12}%
      \getvalue{\s!chemicaltextelement#13}%
      \undochemicaloffset}}

\def\processchemicaltextelement#1#2#3#4#5%
  {\def\doprocess[##1##2##3##4##5]%
     {\doifelse{##1}{?}
        {\doprocess[1..\maxchemical ????]}
        {\doifchemicalnumber{##1}{#1#2}
           {\doifelse{##2##3}{..}
              {\doifchemicalnumber{##4}{#1#2}
                 {\getchemicaltextelement[#1][##1][#4][#5][#3]%
                  \doifnot{##1}{##4}
                    {\!!counta=##1\relax
                     \advance\!!counta by 1
                     \edef\nextsegment{\the\!!counta}%
                     \doprocess[\nextsegment..##4##5]}}}
              {\getchemicaltextelement[#1][##1][#4][#5][#3]%
               \doifnot{##2}{?}{\doprocess[##2##3##4##5]}}}}}%
   \doprocess[#2]%
   \smallchemicaltextfalse}

\def\processchemicalsmalltextelement%
  {\smallchemicaltexttrue\processchemicaltextelement}

\def\processchemicalsmalltextconstant%
  {\smallchemicaltexttrue\processchemicaltextconstant}

\def\processchemicalunrotatedtextelement#1#2#3#4#5#6%
  {\bgroup
   \xdef\@@xxx{0}%
   \xdef\@@yyy{0}%
   \def\putchemicaltext##1##2%
     {\xdef\@@xxx{##1}%
      \xdef\@@yyy{##2}}%
   \getvalue{\s!chemicaltextelement#1}%
   \egroup
   \bgroup
   \def\doputchemicaltext##1 [##2] at ##3 ##4 %
     {\ifnum\chemicaldrawingmode=1
        \rput
          {\chemicalangle}(\chemicalxoffset,\chemicalyoffset)%
          {\rput{-\chemicalangle}(\@@xxx,\@@yyy){\expanded{\rput[##2](##3,##4){##1}}}}%
      \else
        \put
          {\stoprotation \setcoordinatesystem point at 0 0
           \expanded{\put {##1} [##2] at {##3} {##4} }}
          at {\@@xxx} {\@@yyy}
      \fi}%
   \processchemicaltextelement{#2}{#3}{#4}{#5}{#6}%
   \egroup}

\newif\ifchemicaltextconstant

\def\processchemicaltextconstant#1#2#3#4%
  {\chemicaltextconstanttrue
   \let\@@oldchemicalframe=\@@chemicalframe
   \let\@@chemicalframe=\relax
   \processchemicaltextelement{#1}{#2}{#3}{#4}{}%
   \let\@@chemicalframe=\@@oldchemicalframe
   \chemicaltextconstantfalse}

% \plotchemicalline{#1}{#2}{#3}{#4}
%
% #1: x-coordinaat beginpunt
% #2: y-coordinaat beginpunt
% #3: x-coordinaat eindpunt
% #4: y-coordinaat eindpunt

\chardef\chemicallinetype=0

\def\doplotchemicalline%
  {\!!counte=\!!countc \advance\!!counte by -\!!counta
   \!!countf=\!!countd \advance\!!countf by -\!!countb
   \bgroup
   \ifcase\chemicaldrawingmode
     \ifcase\chemicallinetype
     % 0 : normal line
       \plot {\!!counta} {\!!countb} {\!!countc} {\!!countd} /%
     \or
     % 1 : normal arrow
       \arrow <5pt> [.2,.67] from {\!!counta} {\!!countb} to {\!!countc} {\!!countd}
     \or
     % 2 : reverse arrow
       \arrow <5pt> [.2,.67] from {\!!countc} {\!!countd} to {\!!counta} {\!!countb}
     \or
     % 3 : unrotated line
       \put {\stoprotation \setcoordinatesystem point at 0 0
             \plot 0 0 {\!!counte} {\!!countf} /} 
            [\chemicallineposition] at {\!!counta} {\!!countb}
     \else
     % 4 : dashed line
       \findlength {\plot {\!!counta} {\!!countb} {\!!countc} {\!!countd} /}%
       \setdashesnear <2pt> for <\totalarclength>%
       \plot {\!!counta} {\!!countb} {\!!countc} {\!!countd} /%
     \fi
   \or
     \ifcase\chemicallinetype
       \rput
         {\chemicalangle}(\chemicalxoffset,\chemicalyoffset)%
         {\psline(\!!counta,\!!countb)(\!!countc,\!!countd)}%
     \or
       \rput
         {\chemicalangle}(\chemicalxoffset,\chemicalyoffset)%
         {\psline{->}(\!!counta,\!!countb)(\!!countc,\!!countd)}%
     \or
       \rput
         {\chemicalangle}(\chemicalxoffset,\chemicalyoffset)%
         {\psline{<-}(\!!counta,\!!countb)(\!!countc,\!!countd)}%
     \or
       \rput
         {\chemicalangle}(\chemicalxoffset,\chemicalyoffset)%
         {\expanded{\rput[\chemicallineposition]{-\chemicalangle}%
            (\!!counta,\!!countb){\psline(0,0)(\!!counte,\!!countf)}}}%
     \else
       \psset{linestyle=dashed}%
       \rput
         {\chemicalangle}(\chemicalxoffset,\chemicalyoffset)%
         {\psline(\!!counta,\!!countb)(\!!countc,\!!countd)}%
     \fi
   \or
     \global\MPdrawingdonetrue
     \startMPdrawing
        x0 := \MPdivten[\chemicalxoffset]u ;
        y0 := \MPdivten[\chemicalyoffset]u ;
        x1 := \MPdivten[\the\!!counta]u ;
        y1 := \MPdivten[\the\!!countb]u ;
        x2 := \MPdivten[\the\!!countc]u ;
        y2 := \MPdivten[\the\!!countd]u ;
        x3 := \MPdivten[\the\!!counte]u ;
        y3 := \MPdivten[\the\!!countf]u ;
        \ifcase\chemicallinetype
        % 0 : normal line
          draw ((z1--z2) rotatedaround(origin,-\chemicalangle))
            shifted z0 ;
        \or
        % 1 : normal arrow
          drawarrow ((z1--z2) rotatedaround(origin,-\chemicalangle))
            shifted z0 ;
        \or
        % 2 : reverse arrow
          drawarrow ((z2--z1) rotatedaround(origin,-\chemicalangle))
            shifted z0 ;
        \or
        % 3 : unrotated line  % nog \chemicalineposition: t/b
          draw (origin--z3) 
            shifted (z1 rotatedaround(origin,-\chemicalangle))
            shifted z0 ;
        \else
        % 4 : dashed line
          draw ((z1--z2) rotatedaround(origin,-\chemicalangle))
            shifted z0 dashed evenly ;
       \fi
     \stopMPdrawing
   \fi
   \egroup
   \account\!!counta\!!countb
   \account\!!countc\!!countd}

\def\plotchemicalline#1#2#3#4%
  {\!!counta=#1\!!countb=#2\!!countc=#3\!!countd=#4\relax
   \doplotchemicalline}

\def\plotchemicalzline#1#2#3#4%
  {\!!counta=#1\!!countb=#2\!!countc=#3\!!countd=#4\relax
   \ifnum\chemicaldrawingmode=2
     \global\MPdrawingdonetrue
     \startMPdrawing
        x0 := \MPdivten[\chemicalxoffset]u ;
        y0 := \MPdivten[\chemicalyoffset]u ;
        x1 := \MPdivten[\the\!!counta]u ;
        y1 := \MPdivten[\the\!!countb]u ;
        x2 := \MPdivten[\the\!!countc]u ;
        y2 := \MPdivten[\the\!!countd]u ;
        filldraw ((
        \ifnum\chemicalangle>180 
          z1--z2
        \else\ifnum\chemicalangle<90
          z1--(z2 shifted (-2u,+2u))--(z2 shifted (+2u,-2u))
        \else\ifnum\chemicalangle=90
          (z1 shifted (-2u,+2u))--(z1 shifted (+2u,-2u))--
          (z2 shifted (+2u,+2u))--(z2 shifted (-2u,-2u))
        \else
          (z1 shifted (+2u,+2u))--(z1 shifted (-2u,-2u))--z2
        \fi\fi\fi
        --cycle) rotatedaround(origin,-\chemicalangle)) shifted z0 ;
     \stopMPdrawing
   \else
     \doplotchemicalline
     \ifnum\chemicalangle>180 \else
       \ifnum\chemicalangle=90
          \advance\!!counta by -20 \advance\!!countc by -20
          \doplotchemicalline
          \advance\!!counta by  40 \advance\!!countc by  40
        \else\ifnum\chemicalangle<90
          \advance\!!countc by -20 \advance\!!countd by +20
          \doplotchemicalline
          \advance\!!countc by +40 \advance\!!countd by -40
        \else
          \advance\!!counta by  20 \advance\!!countb by  20
          \doplotchemicalline
          \advance\!!counta by -40 \advance\!!countb by -40
        \fi\fi
     \fi
     \doplotchemicalline
   \fi}

\def\plotchemicaldeltaline#1#2#3#4%
  {\!!counta=#1\!!countb=#2\!!countc=#3\!!countd=#4\relax
   \ifnum\chemicaldrawingmode=2
     \global\MPdrawingdonetrue
     \startMPdrawing
        x0 := \MPdivten[\chemicalxoffset]u ;
        y0 := \MPdivten[\chemicalyoffset]u ;
        x1 := \MPdivten[\the\!!counta]u ;
        y1 := \MPdivten[\the\!!countb]u ;
        x2 := \MPdivten[\the\!!countc]u ;
        y2 := \MPdivten[\the\!!countd]u ;
        filldraw ((z1--(z2 rotatedaround(z1,5))--(z2 rotatedaround(z1,-5))
                   --cycle) rotatedaround(origin,-\chemicalangle)) 
          shifted z0 ;
     \stopMPdrawing
     \account{#1}{#2}%
     \account{#3}{#4}%
   \else
     \doplotchemicalline
     \advance\!!countc by  16 \advance\!!countd by -21
     \doplotchemicalline
     \advance\!!countc by  -4 \advance\!!countd by   7
     \doplotchemicalline
     \advance\!!countc by  -4 \advance\!!countd by   7
     \doplotchemicalline
     \advance\!!countc by  -8 \advance\!!countd by  14
     \doplotchemicalline
     \advance\!!countc by  -4 \advance\!!countd by   7
     \doplotchemicalline
     \advance\!!countc by  -4 \advance\!!countd by   7
     \doplotchemicalline
     \advance\!!countc by  -4 \advance\!!countd by   7
     \doplotchemicalline
   \fi}

\def\setchemicallinesegment #1 #2 #3 #4 #5
  {\setvalue{\s!chemicallinesegment#1}{\plotchemicalline{#2}{#3}{#4}{#5}}}

\def\getchemicallinesegment[#1][#2]%
  {\chemicalrotate[#1]%
   \dochemicaloffset{#1}%
   \getvalue{\s!chemicallinesegment#2}%
   \getvalue{\s!chemicallinesegment#21}%
   \getvalue{\s!chemicallinesegment#22}%
   \undochemicaloffset}

\def\getprivatechemicallinesegment[#1][#2]%
  {\chemicalrotate[#1]%
   \getvalue{\s!chemicallinesegment#2#1}}

\def\doprocesschemicallinesegment#1#2#3#4#5%
  {\chardef\chemicallinetype=#1
   \def\chemicallineposition{#2}%
   \def\doprocess[##1##2##3##4##5]%
     {\doifelse{##1}{?}
        {\doprocess[1..\maxchemical ????]}
        {\doifchemicalnumber{##1}{#4#5}
           {\doifelse{##2##3}{..}
              {\doifchemicalnumber{##4}{#4#5}
                 {#3[##1][#4]%
                  \doifnot{##1}{##4}
                    {\!!counta=##1\relax
                     \advance\!!counta by 1
                     \edef\nextsegment{\the\!!counta}%
                     \doprocess[\nextsegment..##4##5]}}}
              {#3[##1][#4]%
               \doifnot{##2}{?}
                 {\doprocess[##2##3##4##5]}}}}}%
   \doprocess[#5]}

\def\processchemicallinesegment%
  {\doprocesschemicallinesegment0c\getchemicallinesegment}

\def\processchemicalzlinesegment#1#2%
  {%\doprocesschemicallinesegment0c\getchemicallinesegment{#1}{#2}%
   \bgroup
   \def\plotchemicalline{\plotchemicalzline}%
   \doprocesschemicallinesegment0c\getchemicallinesegment{#1}{#2}%
   \egroup}

\def\processchemicaldeltalinesegment#1#2%
  {%\doprocesschemicallinesegment0c\getchemicallinesegment{#1}{#2}%
   \bgroup
   \def\plotchemicalline{\plotchemicaldeltaline}%
   \doprocesschemicallinesegment0c\getchemicallinesegment{#1}{#2}%
   \egroup}

\def\processprivatechemicallinesegment%
  {\doprocesschemicallinesegment0c\getprivatechemicallinesegment}

\def\processchemicaldownarrowsegment%
  {\doprocesschemicallinesegment1c\getchemicallinesegment}

\def\processchemicaluparrowsegment%
  {\doprocesschemicallinesegment2c\getchemicallinesegment}

\def\processchemicalunrotatedlinesegment#1%
  {\doprocesschemicallinesegment3{#1}\getchemicallinesegment}

\def\processchemicaldashedlinesegment%
  {\doprocesschemicallinesegment4c\getchemicallinesegment}

\def\processchemicalopenend#1#2%
  {\doprocesschemicallinesegment0c\doprocesschemicalopenend{#1}{#2}}

\def\doprocesschemicalopenend[#1][#2]%
  {\chemicalrotate[#1]%
   \dochemicaloffset{#1}%
   \ifcase\chemicaldrawingmode
     \beginpicture
     \setquadratic\plot
       300 0  400   0
       500 0  550  75
       600 0  650 -75
       700 0  750  75
       800 0  850 -75
       900 0  950   0
             1050   0 /
     \endpicture
   \or
     \rput{\chemicalangle}(\chemicalxoffset,\chemicalyoffset)%
       {\psline(300,0)(500,0)%
        \rput(500,0){\psplot[yunit=75,plotstyle=curve]{0}{720}{x sin}}%
        \psline(950,0)(1050,0)}%
   \or
     \global\MPdrawingdonetrue
     \startMPdrawing
       x0 := \MPdivten[\chemicalxoffset]u ;
       y0 := \MPdivten[\chemicalyoffset]u ;
       draw 
        (((30.0u,0)--(50.0u,0){up}..(55.0u,7.5u)..
          (60.0u,0)..(65.0u,-7.5u)..(70.0u,0)..
          (75.0u,7.5u)..(80.0u,0)..(85.0u,-7.5u)..{up}
          (90.0u,0)--(105.0u,0)) rotatedaround(origin,-\chemicalangle))
        shifted z0 ;
     \stopMPdrawing
   \fi
   \undochemicaloffset}

% \plotchemicalcircle{#1}{#2}{#3}{#4}
%
% #1: lengte van de boog in graden
% #2: x-coordinaat eindpunt
% #3: y-coordinaat eindpunt

\newif\ifchemicaldotted

\def\plotchemicalcircle#1#2#3#4#5#6%
  {\bgroup
   \ifcase\chemicaldrawingmode
     \ifchemicaldotted
       \findlength{\circulararc {#4} degrees from {#5} {#6} center at {0} {0} }%
       \divide\totalarclength by 6
       \def\b{\the\totalarclength}%
       \divide\totalarclength by 2
       \def\a{\the\totalarclength}%
       \setdashpattern <\a,\b,\b,\b,\b,\b,\a>
     \fi
     \circulararc {#4} degrees from {#5} {#6} center at {0} {0} %
   \or
     \ifchemicaldotted
       \psset{linestyle=dashed}%
     \fi
     \rput
       {\chemicalangle}(\chemicalxoffset,\chemicalyoffset)%
       {\psarc(0,0){#3}{#1}{#2}}%
   \or
     \global\MPdrawingdonetrue
     \startMPdrawing
       r := \MPdivten[#3]*2u;
       x0 := \MPdivten[\chemicalxoffset]u ;
       y0 := \MPdivten[\chemicalyoffset]u ;
       draw ((subpath (#1/45,#2/45) of (fullcircle scaled (r))) 
         rotatedaround (origin,\chemicalangle+150))
         shifted z0 \ifchemicaldotted dashed withdots \fi ;
     \stopMPdrawing     
   \fi
   \egroup}

\def\setchemicalcircsegment #1 #2 #3 #4 #5 #6 #7
  {\setvalue{\s!chemicalcircsegment#1}{\plotchemicalcircle{#2}{#3}{#4}{#5}{#6}{#7}}}

\def\getchemicalcircsegment[#1][#2]%
  {\chemicalrotate[#1]%
   \getvalue{\s!chemicalcircsegment#2}}

\def\doprocesschemicalcircsegment#1#2%
  {\def\doprocess[##1##2##3##4##5]%
     {\doifelse{##1}{?}
        {\doprocess[1..\maxchemical ????]}
        {\doifchemicalnumber{##1}{#1#2}
           {\doifelse{##2##3}{..}
              {\doifchemicalnumber{##4}{#1#2}
                 {\getchemicalcircsegment[##1][#1]%
                  \doifnot{##1}{##4}
                    {\!!counta=##1\relax
                     \advance\!!counta by 1
                     \edef\nextsegment{\the\!!counta}%
                     \doprocess[\nextsegment..##4##5]}}}
              {\getchemicalcircsegment[##1][#1]%
               \doifnot{##2}{?}
                 {\doprocess[##2##3##4##5]}}}}}%
   \doprocess[#2]}

\def\processchemicalcircsegment%
  {\chemicaldottedfalse\doprocesschemicalcircsegment}

\def\processchemicaldottsegment%
  {\chemicaldottedtrue\doprocesschemicalcircsegment}

\let\endchemicalpicture   = \relax
\let\checkchemicalpicture = \relax
\let\nomoreaccounting     = \relax

\newif\ifchemicalpicture

\def\beginchemicalpicture#1% NO PSTRICKS SUPPORT YET
  {\checkchemicalpicture
   \bgroup                 % DOES NOT HANDLE AUTOWIDTH/HEIGHT
   \chemicalpicturetrue
   \processchemical[#1]}

\def\setchemicalpicture#1#2%
  {\chemicalpicturefalse
   \def\endchemicalpicture%
     {\@@endchemicallocalpicture{#1}{#2}%
      \egroup
      \ifnum\chemicaldrawingmode=1
        \rput
          {\chemicalangle}(\chemicalxoffset,\chemicalyoffset)%
          {\expanded{\rput[\chemicalloca]{-\chemicalangle}(#1,#2){\box\nextbox}}}%
      \else
        \expanded{\put{\box\nextbox}[\chemicalloca] at {#1} {#2} }
      \fi
      \egroup}%
   \def\checkchemicalpicture%
     {\ifx\endchemicalpicture\relax \else
        \writestatus{ppchtex}{missing end of picture (PE)}%
        \endchemicalpicture
      \fi}%
   \setbox\nextbox=\hbox\bgroup
   \@@beginchemicallocalpicture
   % alternatief: gewoon accounting, en zelf l,r afhandelen
   \ifnum\chemicaldrawingmode=1
     % njet 
   \else
     \accountingon
     \let\nomoreaccounting=\accountingoff
   \fi}

\def\doskipchemical[#1][#2]%
  {{\tt[ppchtex]}}

\def\skipchemical%
  {\dodoubleargument\doskipchemical}

\def\complexchemical% met \expandafter 
  {\ifinchemical
     \expandafter\dochemical
   \else
     \writestatus{ppchtex}{the [][]-alternative is not permitted here}%
     \expandafter\skipchemical
   \fi}

\newif\ifinnerchemical

\def\dosimplechemical#1#2#3%
  {\doifdefinedelse{\??chemical\c!plaats}
     {\writestatus{ppchtex}{the {}{}-alternative is not permitted here}}
     {\ifinnerchemical
        \let\chemicalsign         = \chemicalinnersign
        \let\chemicalmolecule     = \chemicalinnermolecule
        \let\chemicalsinglearrow  = \chemicalsingleinnerarrow
        \let\chemicaldoublearrow  = \chemicaldoubleinnerarrow
        \let\chemicaltwintiparrow = \chemicaltwintipinnerarrow
      \else
        \let\chemicalsign         = \chemicaloutersign
        \let\chemicalmolecule     = \chemicaloutermolecule
        \let\chemicalsinglearrow  = \chemicalsingleouterarrow
        \let\chemicaldoublearrow  = \chemicaldoubleouterarrow
        \let\chemicaltwintiparrow = \chemicaltwintipouterarrow
      \fi
      \disablechemicalspecials
      \unexpandedprocessallactionsinset
        [#1]
        [       HIGH=>\sethighsubscripts,     
                 LOW=>\setlowsubscripts,      
                PLUS=>\chemicalsign{+},
               GIVES=>\chemicalsinglearrow{#2}{#3},
         EQUILIBRIUM=>\chemicaldoublearrow{#2}{#3},
           MESOMERIC=>\chemicaltwintiparrow{#2}{#3},
              SINGLE=>\singlechemicalbond,
              DOUBLE=>\doublechemicalbond,
              TRIPLE=>\triplechemicalbond,
                   +=>\chemicalsign{+},
                  ->=>\chemicalsinglearrow{#2}{#3},
                 <->=>\chemicaldoublearrow{#2}{#3},
                  <>=>\chemicaltwintiparrow{#2}{#3},
                   -=>\singlechemicalbond,
                  --=>\doublechemicalbond,
                 ---=>\triplechemicalbond,
          \s!unknown=>\enablechemicalspecials
                      \chemicalmolecule{\commalistelement}{#2}{#3}]}}

%\def\dosimplechemicalA#1#2#3%
%  {\let\chemicalspace=\relax
%   \def\dodosimplechemical##1%
%     {\dosimplechemical{##1}{}{}}%
%   \@EA\processcommalist\@EA[\@@chemicalchemicaloffset,#1]\dodosimplechemical
%   \egroup}

\def\dosimplechemicalA#1#2#3% % evt: {#1,\relax}
  {\let\chemicalspace=\relax 
   \@EA\dosimplechemical\@EA{\@@chemicalchemicaloffset,#1}{#2}{#3}%
   \egroup}

\def\dosimplechemicalB#1#2#3%
  {\dosimplechemical{#1}{#2}{#3}%
   \egroup}

\def\dosimplechemicalC#1#2#3%
  {$\simplechemical{#1}{#2}{#3}$%
   \egroup} % erbij

\def\simplechemical%
  {\ifinner
     \innerchemicaltrue
   \else
     \innerchemicalfalse
   \fi
   \bgroup
   \catcode`\^=\@@superscript %  t.b.v. \enableduplication
   \catcode`\_=\@@subscript   %  t.b.v. de zekerheid
   \ifmmode
     \ifinnerchemical
       \def\next{\dotriplegroupempty\dosimplechemicalA}%
     \else
       \def\next{\dotriplegroupempty\dosimplechemicalB}%
     \fi
   \else
     \def\next{\dotriplegroupempty\dosimplechemicalC}%
   \fi
   \next}

\definecomplexorsimple\chemical

\def\dogotochemical#1#2%
  {\def\dowithchemical%                       % experiment
     {\localgotochemical{#1}}%                % experiment
   \chemical}                                 % experiment

\def\gotochemical%                            % experiment
  {\dosingleargument\dogotochemical}          % experiment

\def\dododochemical#1[#2][#3]%                % experiment
  {\def\simpledododochemical%                 % experiment
     {#1[#2][#3]}%                            % experiment
   \def\complexdododochemical[##1]%           % experiment
     {\def\dowithchemical%                    % experiment
        {\localthisischemical{#2}}%                % experiment
      #1[#3][##1]}%                           % experiment
   \complexorsimple\dododochemical}          % experiment

\def\dodochemical[#1][#2]%
  {\ignorespaces
   \ifinchemical
     \drawchemical[#1][#2]%
     \ignorespaces
   \else
     \startchemical[\c!plaats=\v!intekst]%
       \drawchemical[#1][#2]%
     \expandafter\stopchemical
   \fi
   \ignorespaces}

\def\dochemical[#1]%
  {\def\simpledochemical%
     {\@@writechemicalstatus{ppchtex}{[#1][]}%
      \dodochemical[#1][]}%
   %
   \def\complexdochemical[##1]%
     {\@@writechemicalstatus{ppchtex}{[#1][##1]}%
      \txtchemical=0%
      \dodochemical[#1][##1]}%
   %
   \def\complexdochemical[##1]%                   % experiment
     {\@@writechemicalstatus{ppchtex}{[#1][##1]}% % experiment
      \txtchemical=0%                             % experiment
      \dododochemical\dodochemical[#1][##1]}%     % experiment
   %
   \complexorsimple\dochemical}

% \processlocalchemicals{#1}
%
% #1: commando's

\def\dodoprocesschemical#1%
  {\processchemical[#1????]}

\def\processlocalchemicals#1%
  {\processcommalist[#1]\dodoprocesschemical}

% \drawchemical[#1][#2]
%
% #1: bindingen enz.
% #2: atomen enz.

\def\localdodochemical[#1][#2]%
  {\@@writechemicalstatus{ppchtex}{[#1][#2]}%
   %\bgroup     % koppelen en afmetingen gaat fout, vandaar:
   \advance\levchemical by 1
   \letvalue{\??chemical\s!unknown\the\levchemical}\unknownchemical
   \setevalue{\??chemical\c!tekst\the\levchemical}{\the\txtchemical}%
   \txtchemical=0
   \dodochemical[#1][#2]%
   \@EA\txtchemical\@EA\csname\??chemical\c!tekst\the\levchemical\endcsname
   \@EA\let\@EA\unknownchemical\csname\??chemical\s!unknown\the\levchemical\endcsname
   \advance\levchemical by -1
   %\egroup
   \ignorespaces}

\def\drawchemical[#1][#2]%
  {\ignorespaces
   \def\dodochemical[##1][##2]%
     {\drawchemical[##1][##2]%
      \ignorespaces}%
   \def\dochemical[##1]%
     {\def\simpledochemical%
        {\@@writechemicalstatus{ppchtex}{[##1][#2]}%
         \dodochemical[##1][#2]%
         \ignorespaces}%
      \def\complexdochemical[####1]%
        {\dododochemical\localdodochemical[##1][####1,#2]}%
     \complexorsimple\dochemical}%
   \doif{\@@chemicalstatus}{\v!start}
     {\doifelse{\chemicalname}{\s!unknown}
        {\getvalue{\s!executechemical\defaultchemical}[#2]}
        {\getvalue{\s!executechemical\chemicalname}[#2]}%
      \def\unknownchemical##1%
        {\processunknownchemical[##1][#2]}%
      \processcommalist[\@@chemicaloffset,#1]\dodoprocesschemical}%
   \ignorespaces}

\def\chemicaloxidation#1#2#3%
  {\chemicaltop
     {\ifnum#20=0
        0%
      \else
        #1\expandafter\uppercase\expandafter{\romannumeral#2}%
      \fi}
     {#3}}

\def\chemicaltfraction{\ifinchemical.60\else.8\fi}
\def\chemicalbfraction{\ifinchemical.45\else.6\fi}
\def\chemicallfraction{\ifinchemical.1\else.1\fi}
\def\chemicalrfraction{\ifinchemical.1\else.1\fi}

\def\chemicaltighttext%
  {\def\chemicaltfraction{\ifinchemical.3\else.6\fi}%
   \def\chemicalbfraction{\ifinchemical.2\else.4\fi}%
   \def\chemicallfraction{\ifinchemical0\else0\fi}%
   \def\chemicalrfraction{\ifinchemical0\else0\fi}}

\def\dochemicaltop#1#2#3#4%
  {\vbox
     {\baselineskip=\chemicaltfraction\baselineskip \lineskip0pt
      \halign
        {#1###2\cr
         $\@@chemicalletter{\scriptscriptstyle#3}$\cr
         $\@@chemicalletter{\@@currentchemicalformat#4}$\cr}}}

\def\dochemicalbottom#1#2#3#4%
  {\vtop
     {\baselineskip=\chemicalbfraction\baselineskip \lineskip0pt
      \halign
        {#1###2\cr
         $\@@chemicalletter{\@@currentchemicalformat#4}$\cr
         $\@@chemicalletter{\scriptscriptstyle#3}$\cr}}}

\def\chemicalleft#1#2%
  {\hbox
     {$\@@chemicalletter{\scriptscriptstyle#1}$%
      $\@@chemicalletter{\@@currentchemicalformat\hskip\chemicallfraction em#2}$}}

\def\chemicalright#1#2%
  {\hbox
     {$\@@chemicalletter{\@@currentchemicalformat#2\hskip\chemicalrfraction em}$%
      $\@@chemicalletter{\scriptscriptstyle#1}$}}

\def\chemicalcentered#1%
  {\setbox0=\hbox{$\@@chemicalletter{\scriptscriptstyle#1}$}%
   \setbox2=\hbox{$\@@chemicalletter{\@@currentchemicalformat C}$}% 
   \dimen0=.5\ht2
   \advance\dimen0 by -.5\ht0
   \advance\dimen0 by \dp0
   \raise\dimen0\box0}

\def\chemicalleftcentered#1#2%
  {\hbox
     {\chemicalcentered{#1}%
      $\@@chemicalletter{\@@currentchemicalformat\hskip\chemicallfraction em#2}$}}

\def\chemicalrightcentered#1#2%
  {\hbox
     {$\@@chemicalletter{\@@currentchemicalformat#2\hskip\chemicalrfraction em}$%
      \chemicalcentered{#1}}}

\def\chemicaltop          {\dochemicaltop    \hss   \hss  }
\def\chemicallefttop      {\dochemicaltop    \relax \hss  }
\def\chemicalrighttop     {\dochemicaltop    \hss   \relax}
\def\chemicalbottom       {\dochemicalbottom \hss   \hss  }
\def\chemicalleftbottom   {\dochemicalbottom \relax \hss  }
\def\chemicalrightbottom  {\dochemicalbottom \hss   \relax}

\def\chemicaltopleft    #1{\chemicalleft {\chemicallefttop   {#1}{}}}
\def\chemicalbottomleft #1{\chemicalleft {\chemicalleftbottom{#1}{}}}
\def\chemicaltopright   #1{\chemicalright{\chemicallefttop   {#1}{}}}
\def\chemicalbottomright#1{\chemicalright{\chemicalleftbottom{#1}{}}}

\def\chemicalsmashedleft#1%
  {\bgroup
   \setbox0=\hbox{$\@@chemicalletter{\@@currentchemicalformat C}$}%
   \setbox2=\hbox{$\@@chemicalletter{\@@currentchemicalformat#1}$}%
   \wd2=\wd0
   \box2
   \egroup}

\def\chemicalsmashedmiddle#1%
  {\bgroup
   \setbox0=\hbox{$\@@chemicalletter{\@@currentchemicalformat C}$}%
   \setbox2=\hbox{$\@@chemicalletter{\@@currentchemicalformat#1}$}%
   \hbox{\hskip-.5\wd2\hskip.5\wd0\box2}
   \egroup}

\def\chemicalsmashedright#1%
  {\bgroup
   \setbox0=\hbox{$\@@chemicalletter{\@@currentchemicalformat C}$}%
   \setbox2=\hbox{$\@@chemicalletter{\@@currentchemicalformat#1}$}%
   \hbox to \wd0{\hskip-\wd2\hskip\wd0\box2}%
   \egroup}

\def\+{\tabalign} % is \long in Plain

\def\chemicalforever#1#2%
  {\bgroup
   \setbox0=\hbox{$\@@chemicalletter{\scriptscriptstyle\hskip-.15em#2}$}%
   \wd0=0pt
   \big#1_{\hskip.1em\box0}%
   \egroup}

\def\disablechemicalspecials%
  {\def\+##1{##1}\def\-##1{##1}%
   \def\[{[}\def\]{]}%
   \def\1{}\def\2{}\def\3{}\def\4{}\def\5{}\def\6{}\def\7{}%
   \def\X{}%
   \def\T{}\def\B{}\def\L{}\def\R{}\def\LC{}\def\RC{}%
   \def\TL{}\def\BL{}\def\TR{}\def\BR{}%
   \def\LT{}\def\LB{}\def\RT{}\def\RB{}%
   \def\SL{}\def\SM{}\def\SR{}}

\def\enablechemicalspecials%
  {\def\+{\dodoublegroupempty\chemicaloxidation{+}}% {} needed!
   \def\-{\dodoublegroupempty\chemicaloxidation{-}}% {} needed!
   \def\[{\dodoublegroupempty\chemicalforever  {[}}% {} needed!
   \def\]{\dodoublegroupempty\chemicalforever  {]}}% {} needed!
   \def\1{\chemicaloxidation\relax1}%
   \def\2{\chemicaloxidation\relax2}%
   \def\3{\chemicaloxidation\relax3}%
   \def\4{\chemicaloxidation\relax4}%
   \def\5{\chemicaloxidation\relax5}%
   \def\6{\chemicaloxidation\relax6}%
   \def\7{\chemicaloxidation\relax7}%
   \def\X{\chemicaltighttext}%
   \def\T{\chemicaltop}%
   \def\B{\chemicalbottom}%
   \def\L{\chemicalleft}%
   \def\LC{\chemicalleftcentered}%
   \def\R{\chemicalright}%
   \def\RC{\chemicalrightcentered}%
   \def\TL{\chemicaltopleft}%
   \def\BL{\chemicalbottomleft}%
   \def\TR{\chemicaltopright}%
   \def\BR{\chemicalbottomright}%
   \def\LT{\chemicallefttop}%
   \def\LB{\chemicalleftbottom}%
   \def\RT{\chemicalrighttop}%
   \def\RB{\chemicalrightbottom}%
   \def\SL{\chemicalsmashedleft}%
   \def\SM{\chemicalsmashedmiddle}%
   \def\SR{\chemicalsmashedright}}

% \reversechemical#1#2#3
%
% #1: prefix
% #2: volgnummer enz
% #3: tegengestelde volgnummers

\def\reversechemical#1#2#3%
  {\def\doprocess[##1##2]%
     {\doifchemicalnumber{##1}{#1#2}%
        {\getfromcommalist[#3][##1]%
         \let\reversechemicalaction=\commalistelement
         \processchemical[#1\reversechemicalaction##2]}}%
   \doprocess[#2]}

% \processunknownchemical[#1????][#2]
%
% #1: bindingen enz.
% #2: atomen enz.

\def\defaultchemical%
  {SIX}

\def\processunknownchemical[#1????][#2]%
  {\processaction
     [#1]
     [       SAVE=>\executechemicalSAVE,
          RESTORE=>\executechemicalRESTORE,
             HIGH=>\sethighsubscripts,
              LOW=>\setlowsubscripts,
      \s!default=>,
      \s!unknown=>\doifdefinedelse{\s!executechemical#1}
                    {\def\chemicalrotation{1}%
                     \def\chemicaloffset{0}%
                     \doifdefined{\s!executechemical#1}
                       {\getvalue{\s!executechemical#1}[#2]}%
                     \@@chemicalpostponed}
                    {\getpredefinedchemical{#1}}]}

\newcount\chemicalstack % tzt \newwounter 

\setvalue{\s!chemical\c!x1}{0}
\setvalue{\s!chemical\c!y1}{0}

\def\executechemicalSAVE%
  {%\writestatus{ppchtex}{saving \the\horchemical,\the\verchemical}%
   \advance\chemicalstack by 1
   \letvalue {\s!chemical n\the\chemicalstack}=\chemicalname
   %\letvalue {\s!chemical p\the\chemicalstack}=\@@chemicalpostponed
   \setevalue{\s!chemical x\the\chemicalstack}{\the\horchemical}%
   \setevalue{\s!chemical y\the\chemicalstack}{\the\verchemical}}

\def\restorechemicalvalues#1%
  {\let\oldprocesschemical=\processchemical
   \doifdefined{\s!executechemical#1}{\getvalue{\s!executechemical#1}[]}%
   \let\processchemical=\oldprocesschemical}

\def\executechemicalRESTORE%
  {\ifnum\chemicalstack=0\relax
     \horchemical=\getvalue{\s!chemical x1}\relax
     \verchemical=\getvalue{\s!chemical y1}\relax
   \else
     \restorechemicalvalues{\getvalue{\s!chemical n\the\chemicalstack}}%
     %\@EA\let\@EA\@@chemicalpostponed\@EA=\csname\s!chemical p\the\chemicalstack\endcsname
     \let\@@chemicalpostponed=\relax
     \horchemical=\getvalue{\s!chemical x\the\chemicalstack}\relax
     \verchemical=\getvalue{\s!chemical y\the\chemicalstack}\relax
     \advance\chemicalstack by -1
   \fi
   \restorechemicalcoordinates}

% De onderstaande macro's zijn verantwoordelijk voor het zetten
% van de + en pijlen. De +, en dus ook de pijlen, worden omhoog
% gehaald. Dit oogt m.i. fraaier.

\def\chemicalraise#1#2%
  {{\setbox0=\hbox{$#1+$}%
    \raise\dp0\hbox{$#1#2$}}}

\def\chemicalinnersign#1%
  {\chemicalraise{\@@localchemicalstyle}{#1}}

\def\chemicaloutersign#1%
  {\chemicalraise{}{#1}}

\def\chemicalsingleinnerarrow#1#2%
  {\chemicalraise{\@@localchemicalstyle}{\longrightarrow}}

\def\chemicaldoubleinnerarrow#1#2%
  {\lower.2ex\hbox
     {\setbox0=\hbox{$\@@localchemicalstyle\longrightarrow$}%
      \setbox2=\hbox{$\@@localchemicalstyle\longleftarrow$}%
      \wd0=0pt\raise\ht0\box0\box2}}

\def\chemicaltwintipinnerarrow#1#2%
  {\hbox
     {\setbox0=\hbox{\chemicalraise{\@@localchemicalstyle}{\longrightarrow}}%
      \setbox2=\hbox{\chemicalraise{\@@localchemicalstyle}{\longleftarrow}}%
      \wd0=0pt\box0\box2}}

\def\dochemicalouterarrow#1#2#3%
  {\bgroup
   \setbox0=\hbox{$\longrightarrow$}%
   \setbox2=\hbox{$\@@chemicalletter{\scriptstyle\quad#2\quad}$}%
   \setbox4=\hbox{$\@@chemicalletter{\scriptstyle\quad#3\quad}$}%
   \dimen2=\wd0 % \dimen0 is used elsewhere
   \ifdim\wd2>\dimen2 \dimen0=\wd2 \fi
   \ifdim\wd4>\dimen2 \dimen0=\wd4 \fi
   \chemicaloutermolecule
     {#1}
     {\ifdim\ht2>\!!zeropoint\box2\fi}  % expands to \empty in test
     {\ifdim\ht4>\!!zeropoint\box4\fi}% % expands to \empty in test
   \egroup}

\def\chemicalsingleouterarrow%
  {\dochemicalouterarrow
     {\hbox to \dimen2{\rightarrowfill}}}

\def\chemicaldoubleouterarrow%
  {\dochemicalouterarrow
     {\lower.5\ht0\vbox
        {\offinterlineskip
         \hbox to \dimen2{\rightarrowfill}
         \hbox to \dimen2{\leftarrowfill}}}}

\def\chemicaltwintipouterarrow%
  {\dochemicalouterarrow
     {\hbox
        {\hbox to \dimen2{\rightarrowfill}%
         \hskip-\dimen2
         \hbox to \dimen2{\leftarrowfill}}}}

\def\chemicalinnermolecule#1#2#3% #3 vergeten 
  {\chemicalspace
   %\mathop % onnodig en genereert spatie aan het begin van een regel
     {\dosetsubscripts
      \hbox{$\@@chemicalletter{\@@localchemicalstyle\strut#1}$}%
      \doresetsubscripts}%
   \chemicalspace}

%\def\chemicaloutermolecule#1#2#3%
%  {\chemicalspace
%   \setbox0=\hbox % else the font is reset 
%     {\dosetsubscripts
%      \hbox{$\@@chemicalletter{\strut#1}$}%
%      \doresetsubscripts}%
%   \mathop{\box0}%
%   \doifnot{#2}{}
%     {_{\@@chemicalletter{\strut\ifthirdargument#3\else#2\fi}}}%
%   \doifnot{#3}{}
%     {^{\@@chemicalletter{\strut\ifthirdargument#2\else#3\fi}}}%
%   \chemicalspace}

\def\chemicaloutermolecule#1#2#3%
  {\chemicalspace
   \setbox0=\hbox % else the font is reset
     {\dosetsubscripts
      \hbox{$\@@chemicalletter{\strut#1}$}%
      \doresetsubscripts}%
   \mathop{\box0}%
   \ifthirdargument
     \doifnot{#2}{}
       {^{\@@chemicalletter{\strut#2}}}%
     \doifnot{#3}{}
       {_{\@@chemicalletter{\strut#3}}}%
   \else
     \doifnot{#2}{}
       {_{\@@chemicalletter{\strut#2}}}%
   \fi
   \chemicalspace}

\def\chemicalsinglepicturearrow#1%
  {\lower.5ex\hbox
      {$\chemicalspace
        \buildrel
          \@@chemicalletter{\scriptstyle\quad#1\quad}%
        \over{\overrightarrow
          {\hphantom{\@chemicalletter{\scriptstyle\quad#1\quad}}}}%
        \chemicalspace$}}

\def\chemicaldoublepicturearrow#1%
  {\lower.5ex\hbox
     {$\chemicalspace
       \buildrel
         \@@chemicalletter{\scriptstyle\quad#1\quad}%
       \over{\overrightarrow{\overleftarrow
         {\hphantom{\@@chemicalletter{\scriptstyle\quad#1\quad}}}}}%
       \chemicalspace$}}

% Bij de in-line bindingen wordt gebruik gemaakt van
% een \hrule. De maatvoering wordt bepaald door een
% kunstmatige em (\wd0).

\def\somechemicalbond%
  {\hrule width \wd0 height .4pt}

\def\dochemicalbonds#1#2#3%
  {{\setbox0=\hbox
      {$\@@chemicalletter{\@@localchemicalstyle M}$}%
    \vbox to \ht0
      {\hsize\wd0
       \vskip.1\wd0#1\vfill#2\vfill#3\vskip.1\wd0}}}

\def\singlechemicalbond%
  {\dochemicalbonds{}{\somechemicalbond}{}}

\def\doublechemicalbond%
  {\dochemicalbonds{\somechemicalbond}{}{\somechemicalbond}}

\def\triplechemicalbond%
  {\dochemicalbonds{\somechemicalbond}{\somechemicalbond}{\somechemicalbond}}

% In plaats van \def\naam{\chemie[...]...} kan beter gebruik
% worden gemaakt van het commando
%
%   \definieerchemie[naam]{commando's}
%
% De naam krijgt, om problemen met bestaande macro's te
% voorkomen, een prefix. Bij het ophalen van een commando
% worden beide definities afgehandeld.

\def\dodefinechemical[#1]#2%
  {\doifdefined{\??chemical#1}
     {\writestatus{ppchtex}{chemical definition #1 is redefined}}%
   \setvalue{\??chemical#1}{#2}}

\def\definechemical%
  {\dosingleargument\dodefinechemical}

\def\getpredefinedchemical#1%
  {\doifdefinedelse{\??chemical#1}
     {\getvalue{\??chemical#1}}
     {\doifdefinedelse{#1}
        {\getvalue{#1}}
        {\writestatus{ppchtex}{unknown chemical definition #1}}}}

% Hieronder zijn de definities van de structuren opgenomen. De
% naam van de structuur is als volgt opgebouwd:
%
%   \executechemicalNUMBER[#1]
%
% waarbij [#1] betrekking heeft op de tekstelementen van \chemie,
% de [tweede lijst] dus.
%
% De aan \chemie[#1][#2] meegegeven lijst van segmenten wordt
% deels door de in \execute gedefinieerde macro's afgehandeld,
% deels door algemene macro's. Segmenten hebben de vorm:
%
%   [+|-|]identifier[X|XYZ|X..Y]
%
% Voorbeelden van segmenten zijn:
%
%   R1
%   R1..4
%   R135
%   -R1
%   +R35

\setchemicalmaximum 0

\def\processchemical[#1]%
  {\unknownchemical{#1}}

\def\setchemicalname#1 %
  {\def\chemicalname{#1}}

\let\chemicalname=\s!unknown

% Vooruitlopend op een gedetailleerde documentatie, zijn hier
% vast enkele gebruikte afmetingen:
%
% lengte radikalen          :  500
% afstand radikalen         :  100
% afstand dubbele radikalen :  260
% afstand substituenten     : +125

\def\executechemicalONE[#1]%
  {\setchemicalname ONE
   %
   \setchemicalmaximum      8
   \setchemicaldistance     0
   \setchemicalsubstitute 625
   \setchemicaldirection  303
   %
   \setchemicalrotation 1  1      0      1      0      1      0      1      0
   \setchemicalrotation 2  0.707 -0.707  0.707 -0.707  0.707 -0.707  0.707 -0.707
   \setchemicalrotation 3  0     -1      0     -1      0     -1      0     -1
   \setchemicalrotation 4 -0.707 -0.707 -0.707 -0.707 -0.707 -0.707 -0.707 -0.707
   \setchemicalrotation 5 -1      0     -1      0     -1      0     -1      0
   \setchemicalrotation 6 -0.707  0.707 -0.707  0.707 -0.707  0.707 -0.707  0.707
   \setchemicalrotation 7  0      1      0      1      0      1      0      1
   \setchemicalrotation 8  0.707  0.707  0.707  0.707  0.707  0.707  0.707  0.707
   %
   \setchemicalangle 1   0  90 180 270
   \setchemicalangle 2  45 135 225 315
   \setchemicalangle 3  90 180 270   0
   \setchemicalangle 4 135 225 315  45
   \setchemicalangle 5 180 270   0  90
   \setchemicalangle 6 225 315  45 135
   \setchemicalangle 7 270   0  90 180
   \setchemicalangle 8 315  45 135 225
   %
   \setchemicaltranslate 1 -1000     0
   \setchemicaltranslate 2 -1000  1000
   \setchemicaltranslate 3     0  1000
   \setchemicaltranslate 4  1000  1000
   \setchemicaltranslate 5  1000     0
   \setchemicaltranslate 6  1000 -1000
   \setchemicaltranslate 7     0 -1000
   \setchemicaltranslate 8 -1000 -1000
   %
   \setchemicallinesegment  SB  300   0 700    0
   \setchemicallinesegment DB1  300  50 700   50
   \setchemicallinesegment DB2  300 -50 700  -50
   %
   \setchemicallinesegment  EP  200 125 200 -125
   \setchemicaltextelement  ES  200   0
   \setchemicaltextelement ED1  200  50
   \setchemicaltextelement ED2  200 -50
   \setchemicaltextelement ET1  200  75
   \setchemicaltextelement ET2  200   0
   \setchemicaltextelement ET3  200 -75
   \setchemicaltextelement HB1  300   0
   \setchemicaltextelement HB2  475   0
   \setchemicaltextelement HB3  650   0
   %
   \setchemicaltextelement   Z  800    0
   \setchemicaltextelement  RZ  950    0
   \setchemicaltextelement  ZN  500    0
   \setchemicaltextelement ZTN  500  150
   \setchemicaltextelement ZBN  500 -150
   %
   \def\processchemical[##1##2##3##4##5]%
     {\doprocesschemical[##1##2##3##4##5]
        {\processaction
           [##1##2##3##4##5]
           [    PB:##4##5=>\beginchemicalpicture{##4##5},
                   PE????=>\endchemicalpicture,
                SUB##4##5=>\processchemicalsubstitute{##4##5},
                ADJ##4##5=>\processchemicaldistance{##4##5},
                MOV##4##5=>\processchemicaltranslate{##4##5},
                DIR##4##5=>\processchemicaldirection{##4##5},
                OFF##4##5=>\processchemicaloffset{##4##5},
CCD##4##5=>\processchemicaldottsegment{CC}{##4##5},
                LDD##4##5=>\processchemicaldashedlinesegment{DB1}{##4##5}%
                           \processchemicallinesegment{DB2}{##4##5},
                RDD##4##5=>\processchemicallinesegment{DB1}{##4##5}%
                           \processchemicaldashedlinesegment{DB2}{##4##5},
                OF##3:##5=>\processchemicalphantom{##3}{##5},
              OE##3##4##5=>\processchemicalopenend{OE}{##3##4##5},
              EP##3##4##5=>\processchemicallinesegment{EP}{##3##4##5},
              ES##3##4##5=>\processchemicaltextconstant{ES}{##3##4##5}{\hbox{$\cdot$}}{0},
              ED##3##4##5=>\processchemicaltextconstant{ED}{##3##4##5}{\hbox{$\cdot$}}{0},
              ET##3##4##5=>\processchemicaltextconstant{ET}{##3##4##5}{\hbox{$\cdot$}}{0},
              HB##3##4##5=>\processchemicaltextconstant{HB}{##3##4##5}{\hbox{$\cdot$}}{0},
              SB##3##4##5=>\processchemicallinesegment{SB}{##3##4##5},
              DB##3##4##5=>\processchemicallinesegment{DB}{##3##4##5},
              SB##3##4##5=>\processchemicallinesegment{SB}{##3##4##5},
              BB##3##4##5=>\processchemicaldeltalinesegment{SB}{##3##4##5},
              SD##3##4##5=>\processchemicaldashedlinesegment{SB}{##3##4##5},
              TB##3##4##5=>\processchemicallinesegment{SB}{##3##4##5}%
                           \processchemicallinesegment{DB}{##3##4##5},
              CZ##3##4##5=>\processchemicaltextelement{RZ}{##3##4##5}{#1}{0}{},
                ZTN##4##5=>\processchemicalsmalltextconstant{ZTN}{##4##5}{\chemicaltextelementnumber}{0},
                ZTT##4##5=>\processchemicalsmalltextelement{ZTN}{##4##5}{#1}{0}{},
                ZBN##4##5=>\processchemicalsmalltextconstant{ZBN}{##4##5}{\chemicaltextelementnumber}{0},
                ZBT##4##5=>\processchemicalsmalltextelement{ZBN}{##4##5}{#1}{0}{},
              ZN##3##4##5=>\processchemicaltextconstant{ZN}{##3##4##5}{\chemicaltextelementnumber}{0},
              ZT##3##4##5=>\processchemicaltextelement{ZN}{##3##4##5}{#1}{0}{},
            Z##2##3##4##5=>\processchemicaltextelement{Z}{##2##3##4##5}{#1}{0}
                             {l,l,t,r,r,r,b,l},
            \s!unknown=>\unknownchemical{##1##2##3##4##5}]}}}

\def\executechemicalTHREE[#1]%
  {\setchemicalname THREE
   %
   \setchemicalmaximum      3
   \setchemicaldistance   289
   \setchemicalsubstitute 952
   %
   \setchemicalrotation 1  1    0      0     -1   -1    0      0      1
   \setchemicalrotation 2 -0.5 -0.866 -0.866  0.5  0.5  0.866  0.866 -0.5
   \setchemicalrotation 3 -0.5  0.866  0.866  0.5  0.5 -0.866 -0.866 -0.5
   %
   \setchemicalangle 1   0  90 180 270
   \setchemicalangle 2 120 210 300  30
   \setchemicalangle 3 240 330  60 150
   %
   \setchemicaltranslate 1 -1000     0
   \setchemicaltranslate 2     0  1000
   \setchemicaltranslate 3  1000     0
   \setchemicaltranslate 4     0 -1000
   %
   \setchemicallinesegment   B   577     0  -289  -500
   \setchemicallinesegment  SB   352  -130   -64  -370
   \setchemicallinesegment -SB   352  -130  -289  -500
   \setchemicallinesegment +SB   577     0   -64  -370
   \setchemicallinesegment DB1   327   -87   -89  -327
   \setchemicallinesegment DB2   377  -172   -39  -413
   \setchemicallinesegment   R   577     0  1077     0
   \setchemicallinesegment  -R   577     0  1010   250
   \setchemicallinesegment  +R   577     0  1010  -250
   \setchemicallinesegment ER1   577    50  1077    50
   \setchemicallinesegment ER2   577   -50  1077   -50
   \setchemicallinesegment  SR   837     0  1077     0
   \setchemicallinesegment -SR   802   130  1010   250
   \setchemicallinesegment +SR   802  -130  1010  -250
   \setchemicallinesegment DR1   837    50  1077    50
   \setchemicallinesegment DR2   837   -50  1077   -50
   %
   \setchemicaltextelement   Z   577     0
   \setchemicaltextelement  RZ  1177     0
   \setchemicaltextelement -RZ  1097   300
   \setchemicaltextelement +RZ  1097  -300
   \setchemicaltextelement CRZ  1077     0
   %
   \def\processchemical[##1##2##3##4]%
     {\processaction
        [##1##2##3##4]
        [    ROT##4=>\processchemicalrotation{##4},
             MOV##4=>\processchemicaltranslate{##4},
             SUB##4=>\processchemicalsubstitute{##4},
             ADJ##4=>\processchemicaldistance{##4},
             -RZ##4=>\processchemicaltextelement{-RZ}{##4}{#1}{3}
                       {l,t,r, l,r,l, r,b,l, r,l,r},
             +RZ##4=>\processchemicaltextelement{+RZ}{##4}{#1}{3}
                       {l,r,b, r,r,l, r,l,t, l,l,r},
             -SB##4=>\processchemicallinesegment{-SB}{##4},
             +SB##4=>\processchemicallinesegment{+SB}{##4},
             -SR##4=>\processchemicallinesegment{-SR}{##4},
             +SR##4=>\processchemicallinesegment{+SR}{##4},
             CRZ##4=>\processchemicaltextelement{CRZ}{##4}{#1}{0}
                       {},
           DB##3##4=>\processchemicallinesegment{DB}{##3##4},
           DR##3##4=>\processchemicallinesegment{DR}{##3##4},
           RZ##3##4=>\processchemicaltextelement{RZ}{##3##4}{#1}{3}
                       {l,r,r, t,r,l, r,l,l, b,l,r},
           ER##3##4=>\processchemicallinesegment{ER}{##3##4},
           SB##3##4=>\processchemicallinesegment{SB}{##3##4},
           SR##3##4=>\processchemicallinesegment{SR}{##3##4},
           -R##3##4=>\processchemicallinesegment{-R}{##3##4},
           +R##3##4=>\processchemicallinesegment{+R}{##3##4},
         B##2##3##4=>\processchemicallinesegment{B}{##2##3##4},
         R##2##3##4=>\processchemicallinesegment{R}{##2##3##4},
         Z##2##3##4=>\processchemicaltextelement{Z}{##2##3##4}{#1}{0}{},
         \s!unknown=>\unknownchemical{##1##2##3##4}]}}

\def\executechemicalFOUR[#1]%
  {\setchemicalname FOUR
   %
   \setchemicalmaximum      4
   \setchemicaldistance   500
   \setchemicalsubstitute   0
   %
   \setchemicalrotation 1  1  0  0 -1 -1  0  0  1
   \setchemicalrotation 2  0 -1 -1  0  0  1  1  0
   \setchemicalrotation 3 -1  0  0  1  1  0  0 -1
   \setchemicalrotation 4  0  1  1  0  0 -1 -1  0
   %
   \setchemicalangle 1   0  90 180 270
   \setchemicalangle 2  90 180 270   0
   \setchemicalangle 3 180 270   0  90
   \setchemicalangle 4 270   0  90 180
   %
   \setchemicaltranslate 1 -1000     0
   \setchemicaltranslate 2     0  1000
   \setchemicaltranslate 3  1000     0
   \setchemicaltranslate 4     0 -1000
   %
   \setchemicallinesegment   B  500  500  500 -500
   \setchemicallinesegment  SB  500  240  500 -240
   \setchemicallinesegment -SB  500  240  500 -500
   \setchemicallinesegment +SB  500  500  500 -240
   \setchemicallinesegment DB1  450  240  450 -240
   \setchemicallinesegment DB2  550  240  550 -240
   \setchemicallinesegment  EB  360  300  360 -300
   \setchemicallinesegment   R  500  500  854  854
   \setchemicallinesegment  -R  500  500  500 1000
   \setchemicallinesegment  +R  500  500 1000  500
   \setchemicallinesegment ER1  465  535  819  889
   \setchemicallinesegment ER2  535  465  889  819
   \setchemicallinesegment  SR  684  684  854  854
   \setchemicallinesegment -SR  500  760  500 1000
   \setchemicallinesegment +SR  760  500 1000  500
   \setchemicallinesegment DR1  649  719  819  889
   \setchemicallinesegment DR2  719  649  889  819
   %
   \setchemicaltextelement   Z  500  500
   \setchemicaltextelement  RZ  925  925
   \setchemicaltextelement -RZ  500 1100
   \setchemicaltextelement +RZ 1100  500
   \setchemicaltextelement CRZ 1038 1038
   %
   \setchemicaltextelement  ZN  350  350
   %
   \def\processchemical[##1##2##3##4]%
     {\processaction
        [##1##2##3##4]
        [    PB:##4=>\beginchemicalpicture{##4},
             PE????=>\endchemicalpicture,
             ROT##4=>\processchemicalrotation{##4},
             SUB##4=>\processchemicalsubstitute{##4},
             ADJ##4=>\processchemicaldistance{##4},
             MOV##4=>\processchemicaltranslate{##4},
             -RZ##4=>\processchemicaltextelement{-RZ}{##4}{#1}{4}
                       {b,l,t,r, l,t,r,b, t,r,b,l, r,b,l,t},
             +RZ##4=>\processchemicaltextelement{+RZ}{##4}{#1}{4}
                       {l,t,r,b, t,r,b,l, r,b,l,t, b,l,t,r},
             -SB##4=>\processchemicallinesegment{-SB}{##4},
             +SB##4=>\processchemicallinesegment{+SB}{##4},
             -SR##4=>\processchemicallinesegment{-SR}{##4},
             +SR##4=>\processchemicallinesegment{+SR}{##4},
             CRZ##4=>\processchemicaltextelement{CRZ}{##4}{#1}{0}{},
           ZN##3##4=>\processchemicaltextconstant{ZN}{##3##4}{\chemicaltextelementnumber}{0},
           ZT##3##4=>\processchemicaltextelement{ZN}{##3##4}{#1}{0}{},
           DB##3##4=>\processchemicallinesegment{DB}{##3##4},
           DR##3##4=>\processchemicallinesegment{DR}{##3##4},
           EB##3##4=>\processchemicallinesegment{EB}{##3##4},
           ER##3##4=>\processchemicallinesegment{ER}{##3##4},
           RZ##3##4=>\processchemicaltextelement{RZ}{##3##4}{#1}{4}
                       {lb,lt,rt,rb, lt,rt,rb,lb, rt,rb,lb,lt, rb,lb,lt,rt},
           SB##3##4=>\processchemicallinesegment{SB}{##3##4},
           SR##3##4=>\processchemicallinesegment{SR}{##3##4},
           -R##3##4=>\processchemicallinesegment{-R}{##3##4},
           +R##3##4=>\processchemicallinesegment{+R}{##3##4},
         B##2##3##4=>\processchemicallinesegment{B}{##2##3##4},
         R##2##3##4=>\processchemicallinesegment{R}{##2##3##4},
         Z##2##3##4=>\processchemicaltextelement{Z}{##2##3##4}{#1}{0}{},
         \s!unknown=>\unknownchemical{##1##2##3##4}]}}

\def\executechemicalFIVE[#1]%
  {\setchemicalname FIVE
   %
   \setchemicalmaximum       5
   \setchemicaldistance    688
   \setchemicalsubstitute 1226
   %
   \setchemicalrotation 1  1      0      0     -1     -1      0      0      1
   \setchemicalrotation 2  0.309 -0.951 -0.951 -0.309 -0.309  0.940  0.951  0.309
   \setchemicalrotation 3 -0.809 -0.588 -0.588  0.809  0.809  0.588  0.588 -0.809
   \setchemicalrotation 4 -0.809  0.588  0.588  0.809  0.809 -0.588 -0.588 -0.809
   \setchemicalrotation 5  0.309  0.951  0.951 -0.309 -0.309 -0.951 -0.951  0.309
   %
   \setchemicalangle 1   0  90 180 270
   \setchemicalangle 2  72 162 252 342
   \setchemicalangle 3 144 234 324  54
   \setchemicalangle 4 216 306  36 126
   \setchemicalangle 5 288  18 108 198
   %
   \setchemicaltranslate 1 -1376     0
   \setchemicaltranslate 2  -425  1304
   \setchemicaltranslate 3  1113   809
   \setchemicaltranslate 4  1113  -809
   \setchemicaltranslate 5  -425 -1304
   %
   \setchemicallinesegment   A 1188  500 1188 -500
   \setchemicallinesegment   B  688  500  688 -500
   \setchemicallinesegment   S -263  808  688 -500
   \setchemicallinesegment  SS -116  606  541 -298
   \setchemicallinesegment -SS -263  808  541 -298
   \setchemicallinesegment +SS -116  606  688 -500
   \setchemicallinesegment  SB  688  240  688 -240
   \setchemicallinesegment -SB  688  240  688 -500
   \setchemicallinesegment +SB  688  500  688 -240
   \setchemicallinesegment DB1  638  240  638 -240
   \setchemicallinesegment DB2  738  240  738 -240
   \setchemicallinesegment  EB  548  340  548 -340
   \setchemicallinesegment   R  688  500 1093  794
   \setchemicallinesegment  -R  688  500  688 1000
   \setchemicallinesegment  +R  688  500 1163  345
   \setchemicallinesegment ER1  659  540 1064  834
   \setchemicallinesegment ER2  727  460 1122  754
   \setchemicallinesegment  SR  898  653 1093  794
   \setchemicallinesegment -SR  688  760  688 1000
   \setchemicallinesegment +SR  935  420 1163  345
   \setchemicallinesegment DR1  869  693 1064  834
   \setchemicallinesegment DR2  927  613 1122  754
   %
   \setchemicaltextelement   Z  688  500
   \setchemicaltextelement  RZ 1188  863
   \setchemicaltextelement -RZ  688 1100
   \setchemicaltextelement +RZ 1258  315
   \setchemicaltextelement CRZ 1323  947
   %
   \setchemicalcircsegment   C -36 36 590 72 475 -345
   \setchemicalcircsegment  CC -72  0 590 72 182 -561
   %
   \setchemicaltextelement  ZN  468  350
   \setchemicaltextelement  RN  860  625 % 1.25 Z
   \setchemicaltextelement  RTN 785  728 % .12 / 103 75
   \setchemicaltextelement  RBN 935  522  
   %
   \def\processchemical[##1##2##3##4]%
     {\processaction
        [##1##2##3##4]
        [ FRONT????=>{\executechemicalFIVEFRONT[#1]},
             PB:##4=>\beginchemicalpicture{##4},
             PE????=>\endchemicalpicture,
             ROT##4=>\processchemicalrotation{##4},
             SUB##4=>\processchemicalsubstitute{##4},
             ADJ##4=>\processchemicaldistance{##4},
             MOV##4=>\processchemicaltranslate{##4},
             -RZ##4=>\processchemicaltextelement{-RZ}{##4}{#1}{5}
                       {b,l,t,r,r, l,t,r,r,l, t,r,r,l,l, r,b,l,t,r},
             +RZ##4=>\processchemicaltextelement{+RZ}{##4}{#1}{5}
                       {l,t,r,r,b, t,r,r,l,l, r,r,l,l,r, b,l,l,r,r},
             -SB##4=>\processchemicallinesegment{-SB}{##4},
             +SB##4=>\processchemicallinesegment{+SB}{##4},
             -SR##4=>\processchemicallinesegment{-SR}{##4},
             +SR##4=>\processchemicallinesegment{+SR}{##4},
             -RD##4=>\processchemicaldashedlinesegment{-R}{##4},
             +RD##4=>\processchemicaldashedlinesegment{+R}{##4},
             -RB##4=>\processchemicaldeltalinesegment{-R}{##4},
             +RB##4=>\processchemicaldeltalinesegment{+R}{##4},
             CRZ##4=>\processchemicaltextelement{CRZ}{##4}{#1}{0}{},
RTN##4=>\processchemicaltextconstant{RTN}{##4}{\chemicaltextelementnumber}{0},
RTT##4=>\processchemicaltextelement{RTN}{##4}{#1}{0}{},
RBN##4=>\processchemicaltextconstant{RBN}{##4}{\chemicaltextelementnumber}{0},
RBT##4=>\processchemicaltextelement{RBN}{##4}{#1}{0}{},
             -SS##4=>\processchemicallinesegment{-SS}{##4},
             +SS##4=>\processchemicallinesegment{+SS}{##4},
             CCD##4=>\processchemicaldottsegment{CC}{##4},
           SS##3##4=>\processchemicallinesegment{SS}{##3##4},
           RD##3##4=>\processchemicaldashedlinesegment{R}{##3##4},
           RB##3##4=>\processchemicaldeltalinesegment{R}{##3##4},
           ZN##3##4=>\processchemicaltextconstant{ZN}{##3##4}{\chemicaltextelementnumber}{0},
           ZT##3##4=>\processchemicaltextelement{ZN}{##3##4}{#1}{0}{},
RN##3##4=>\processchemicaltextconstant{RN}{##3##4}{\chemicaltextelementnumber}{0},
RT##3##4=>\processchemicaltextelement{RN}{##3##4}{#1}{0}{},
           AU##3##4=>\processchemicaluparrowsegment{A}{##3##4},
           AD##3##4=>\processchemicaldownarrowsegment{A}{##3##4},
           CC##3##4=>\processchemicalcircsegment{CC}{##3##4},
           CD##3##4=>\processchemicaldottsegment{C}{##3##4},
           DB##3##4=>\processchemicallinesegment{DB}{##3##4},
           DR##3##4=>\processchemicallinesegment{DR}{##3##4},
           EB##3##4=>\processchemicallinesegment{EB}{##3##4},
           ER##3##4=>\processchemicallinesegment{ER}{##3##4},
           RZ##3##4=>\processchemicaltextelement{RZ}{##3##4}{#1}{5}
                       {l,l,r,r,r, l,r,r,b,l, r,r,b,l,t, r,l,l,t,r},
           SB##3##4=>\processchemicallinesegment{SB}{##3##4},
           SR##3##4=>\processchemicallinesegment{SR}{##3##4},
           -R##3##4=>\processchemicallinesegment{-R}{##3##4},
           +R##3##4=>\processchemicallinesegment{+R}{##3##4},
         B##2##3##4=>\processchemicallinesegment{B}{##2##3##4},
         C##2##3##4=>\processchemicalcircsegment{C}{##2##3##4},
         R##2##3##4=>\processchemicallinesegment{R}{##2##3##4},
         S##2##3##4=>\processchemicallinesegment{S}{##2##3##4},
         Z##2##3##4=>\processchemicaltextelement{Z}{##2##3##4}{#1}{0}{},
         \s!unknown=>\unknownchemical{##1##2##3##4}]}}

\def\executechemicalSIX[#1]%
  {\setchemicalname SIX
   %
   \setchemicalmaximum       6
   \setchemicalsubstitute 1375
   \setchemicaldistance    866
   %
   \setchemicalrotation 1  1    0   0 -1     -1    0    0      1
   \setchemicalrotation 2  0.5 -0.866 -0.866 -0.5 -0.5  0.866  0.866  0.5
   \setchemicalrotation 3 -0.5 -0.866 -0.866  0.5  0.5  0.866  0.866 -0.5
   \setchemicalrotation 4 -1    0      0      1    1    0      0     -1
   \setchemicalrotation 5 -0.5  0.866  0.866  0.5  0.5 -0.866 -0.866 -0.5
   \setchemicalrotation 6  0.5  0.866  0.866 -0.5 -0.5 -0.866 -0.866  0.5
   %
   \setchemicalangle 1   0  90 180 270
   \setchemicalangle 2  60 150 240 330
   \setchemicalangle 3 120 210 300  30
   \setchemicalangle 4 180 270   0  90
   \setchemicalangle 5 240 330  60 150
   \setchemicalangle 6 300  30 120 210
   %
   \setchemicaltranslate 1 -1732     0
   \setchemicaltranslate 2  -866  1500
   \setchemicaltranslate 3   866  1500
   \setchemicaltranslate 4  1732     0
   \setchemicaltranslate 5   866 -1500
   \setchemicaltranslate 6  -866 -1500
   %
   \setchemicallinesegment    A 1386   500 1386 -500
   \setchemicallinesegment    S    0  1000  866 -500
   \setchemicallinesegment   SS  125   783  741 -283
   \setchemicallinesegment  -SS    0  1000  741 -283
   \setchemicallinesegment  +SS  125   783  866 -500
   \setchemicallinesegment    B  866   500  866 -500
   \setchemicallinesegment   SB  866   240  866 -240
   \setchemicallinesegment  -SB  866   240  866 -500
   \setchemicallinesegment  +SB  866   500  866 -240
   \setchemicallinesegment  DB1  816   240  816 -240
   \setchemicallinesegment  DB2  916   240  916 -240
   \setchemicallinesegment   EB  726   340  726 -340
   \setchemicallinesegment    R  866   500 1299  750
   \setchemicallinesegment   -R  866   500  866 1000
   \setchemicallinesegment   +R  866   500 1299  250
   \setchemicallinesegment  ER1  841   543 1274  793
   \setchemicallinesegment  ER2  891   457 1324  707
   \setchemicallinesegment   SR 1091   630 1299  750
   \setchemicallinesegment  -SR  866   740  866 1000
   \setchemicallinesegment  +SR 1091   370 1299  250
   \setchemicallinesegment  DR1 1066   673 1274  793
   \setchemicallinesegment  DR2 1116   588 1324  707
   \setchemicallinesegment MID1    0  1000 -150  200
   \setchemicallinesegment MID2    0 -1000 -150 -200
   \setchemicallinesegment MIDS1   0  1000 -180    0
   \setchemicallinesegment MIDS2   0 -1000 -180    0
   %
   \setchemicalcircsegment   C -30 30 700 60 600 -346
   \setchemicalcircsegment  CC -60  0 700 60 350 -606
   %
   \setchemicaltextelement    Z  866  500
   \setchemicaltextelement   RZ 1386  800
   \setchemicaltextelement  -RZ  866 1100
   \setchemicaltextelement  +RZ 1386  200
   \setchemicaltextelement  CRZ 1524  880
   \setchemicaltextelement MIDZ -150    0
   %
   \setchemicaltextelement  ZN   589 350
   \setchemicaltextelement  RN  1083 625  % 1.25 Z
   \setchemicaltextelement  RTN 1008 755  % .12 / 130 75 
   \setchemicaltextelement  RBN 1158 495   
   %
   \def\processchemical[##1##2##3##4##5]%
     {\processaction
        [##1##2##3##4##5]
        [    FRONT????=>{\executechemicalSIXFRONT[#1]},
               MID????=>\processchemicallinesegment{MID}{1????},
              MIDS????=>\processchemicallinesegment{MIDS}{1????},
              MIDZ????=>\processchemicaltextelement{MIDZ}{1????}{#1}{0}{},
             PB:##4##5=>\beginchemicalpicture{##4##5},
                PE????=>\endchemicalpicture,
             ROT##4##5=>\processchemicalrotation{##4##5},
             SUB##4##5=>\processchemicalsubstitute{##4##5},
             ADJ##4##5=>\processchemicaldistance{##4##5},
             MOV##4##5=>\processchemicaltranslate{##4##5},
             -RZ##4##5=>\processchemicaltextelement{-RZ}{##4##5}{#1}{6}
                       {b,l,l,t,r,r, l,l,r,r,r,l, t,r,r,b,l,l, r,r,l,l,l,r},
             +RZ##4##5=>\processchemicaltextelement{+RZ}{##4##5}{#1}{6}
                       {l,t,r,r,b,l, r,r,r,l,l,l, r,b,l,l,t,r, l,l,l,r,r,r},
             -SB##4##5=>\processchemicallinesegment{-SB}{##4##5},
             +SB##4##5=>\processchemicallinesegment{+SB}{##4##5},
             -SR##4##5=>\processchemicallinesegment{-SR}{##4##5},
             +SR##4##5=>\processchemicallinesegment{+SR}{##4##5},
             -RD##4##5=>\processchemicaldashedlinesegment{-R}{##4##5},
             +RD##4##5=>\processchemicaldashedlinesegment{+R}{##4##5},
             -RB##4##5=>\processchemicaldeltalinesegment{-R}{##4##5},
             +RB##4##5=>\processchemicaldeltalinesegment{+R}{##4##5},
             CRZ##4##5=>\processchemicaltextelement{CRZ}{##4##5}{#1}{0}{},
             -SS##4##5=>\processchemicallinesegment{-SS}{##4##5},
             +SS##4##5=>\processchemicallinesegment{+SS}{##4##5},
             CCD##4##5=>\processchemicaldottsegment{CC}{##4##5},
RTN##4##5=>\processchemicaltextconstant{RTN}{##4##5}{\chemicaltextelementnumber}{0},
RTT##4##5=>\processchemicaltextelement{RTN}{##4##5}{#1}{0}{},
RBN##4##5=>\processchemicaltextconstant{RBN}{##4##5}{\chemicaltextelementnumber}{0},
RBT##4##5=>\processchemicaltextelement{RBN}{##4##5}{#1}{0}{},
           SS##3##4##5=>\processchemicallinesegment{SS}{##3##4##5},
           RD##3##4##5=>\processchemicaldashedlinesegment{R}{##3##4##5},
           RB##3##4##5=>\processchemicaldeltalinesegment{R}{##3##4##5},
           ZN##3##4##5=>\processchemicaltextconstant{ZN}{##3##4##5}{\chemicaltextelementnumber}{0},
           ZT##3##4##5=>\processchemicaltextelement{ZN}{##3##4##5}{#1}{0}{},
RN##3##4##5=>\processchemicaltextconstant{RN}{##3##4##5}{\chemicaltextelementnumber}{0},
RT##3##4##5=>\processchemicaltextelement{RN}{##3##4##5}{#1}{0}{},
           AU##3##4##5=>\processchemicaluparrowsegment{A}{##3##4##5},
           AD##3##4##5=>\processchemicaldownarrowsegment{A}{##3##4##5},
           CD##3##4##5=>\processchemicaldottsegment{C}{##3##4##5},
           CC##3##4##5=>\processchemicalcircsegment{CC}{##3##4##5},
           DB##3##4##5=>\processchemicallinesegment{DB}{##3##4##5},
           EB##3##4##5=>\processchemicallinesegment{EB}{##3##4##5},
           ER##3##4##5=>\processchemicallinesegment{ER}{##3##4##5},
           RZ##3##4##5=>\processchemicaltextelement{RZ}{##3##4##5}{#1}{6}
                       {l,l,t,r,r,b, l,r,r,r,l,l, r,r,b,l,l,t, r,l,l,l,r,r},
           SB##3##4##5=>\processchemicallinesegment{SB}{##3##4##5},
           SR##3##4##5=>\processchemicallinesegment{SR}{##3##4##5},
           DR##3##4##5=>\processchemicallinesegment{DR}{##3##4##5},
           -R##3##4##5=>\processchemicallinesegment{-R}{##3##4##5},
           +R##3##4##5=>\processchemicallinesegment{+R}{##3##4##5},
         B##2##3##4##5=>\processchemicallinesegment{B}{##2##3##4##5},
         C##2##3##4##5=>\processchemicalcircsegment{C}{##2##3##4##5},
         R##2##3##4##5=>\processchemicallinesegment{R}{##2##3##4##5},
         S##2##3##4##5=>\processchemicallinesegment{S}{##2##3##4##5},
         Z##2##3##4##5=>\processchemicaltextelement{Z}{##2##3##4##5}{#1}{0}{},
            \s!unknown=>\unknownchemical{##1##2##3##4##5}]}}

\def\executechemicalSEVEN[#1]% incomplete ! 
  {\setchemicalname SEVEN
   %
   \setchemicalmaximum        7
   \setchemicalsubstitute     -
   \setchemicaldistance    1038
   %
   \setchemicalrotation 1  .623  .782 - - - - - -
   \setchemicalrotation 2 -.223  .975 - - - - - -
   \setchemicalrotation 3 -.901  .434 - - - - - -
   \setchemicalrotation 4 -.901 -.434 - - - - - -
   \setchemicalrotation 5 -.223 -.975 - - - - - -
   \setchemicalrotation 6  .623 -.782 - - - - - -
   \setchemicalrotation 7     1     0 - - - - - -
   %
   \setchemicalangle 1       0 - - -
   \setchemicalangle 2  51.429 - - -
   \setchemicalangle 3 102.857 - - -
   \setchemicalangle 4 154.286 - - -
   \setchemicalangle 5 205.714 - - -
   \setchemicalangle 6 257.143 - - -
   \setchemicalangle 7 308.571 - - -
   %
   \setchemicaltranslate 1 - - 
   \setchemicaltranslate 2 - -
   \setchemicaltranslate 3 - -
   \setchemicaltranslate 4 - -
   \setchemicaltranslate 5 - -
   \setchemicaltranslate 6 - -
   \setchemicaltranslate 7 - -
   %
   \setchemicallinesegment   B 1038 500 1038 -500
   \setchemicallinesegment  SB 1038 240 1038 -240
   \setchemicallinesegment -SB 1038 240 1038 -500
   \setchemicallinesegment +SB 1038 500 1038 -240
   %
   \setchemicaltextelement Z 1038 500
   %
   \def\processchemical[##1##2##3##4##5]%
     {\processaction
        [##1##2##3##4##5]
        [    PB:##4##5=>\beginchemicalpicture{##4##5},
                PE????=>\endchemicalpicture,
            %ROT##4##5=>\processchemicalrotation{##4},
            %SUB##4##5=>\processchemicalsubstitute{##4##5},
            %ADJ##4##5=>\processchemicaldistance{##4##5},
            %MOV##4##5=>\processchemicaltranslate{##4##5},
             -SB##4##5=>\processchemicallinesegment{-SB}{##4##5},
             +SB##4##5=>\processchemicallinesegment{+SB}{##4##5},
           SB##3##4##5=>\processchemicallinesegment{SB}{##3##4##5},
         B##2##3##4##5=>\processchemicallinesegment{B}{##2##3##4##5},
         Z##2##3##4##5=>\processchemicaltextelement{Z}{##2##3##4##5}{#1}{0}{},
            \s!unknown=>\unknownchemical{##1##2##3##4##5}]}}

\def\executechemicalEIGHT[#1]% incomplete !
  {\setchemicalname EIGHT
   %
   \setchemicalmaximum       8
   %\setchemicalsubstitute 1307
   \setchemicaldistance    1207
   %
   \setchemicalrotation 1  .707  .707 - - - - - -
   \setchemicalrotation 2     0     1 - - - - - -
   \setchemicalrotation 3 -.707  .707 - - - - - -
   \setchemicalrotation 4    -1     0 - - - - - -
   \setchemicalrotation 5 -.707 -.707 - - - - - -
   \setchemicalrotation 6    0     -1 - - - - - -
   \setchemicalrotation 7  .707 -.707 - - - - - -
   \setchemicalrotation 8     1     0 - - - - - -
   %
   \setchemicalangle 1  45 - - -
   \setchemicalangle 2  90 - - -
   \setchemicalangle 3 135 - - -
   \setchemicalangle 4 180 - - -
   \setchemicalangle 5 225 - - -
   \setchemicalangle 6 270 - - -
   \setchemicalangle 7 315 - - -
   \setchemicalangle 8   0 - - -
   %
   \setchemicaltranslate 1 -2414     0
   \setchemicaltranslate 2 -1706  1706
   \setchemicaltranslate 3     0  2414
   \setchemicaltranslate 4  1706  1706
   \setchemicaltranslate 5  2414     0
   \setchemicaltranslate 6  1706 -1706
   \setchemicaltranslate 7     0 -2414
   \setchemicaltranslate 8 -1706 -1706
   %
   \setchemicallinesegment    B 1207 500 1207 -500
   \setchemicallinesegment   SB 1207 240 1207 -240
   \setchemicallinesegment  -SB 1207 240 1207 -500
   \setchemicallinesegment  +SB 1207 500 1207 -240
   %
   \setchemicaltextelement    Z 1207 500
   %
   \def\processchemical[##1##2##3##4##5]%
     {\processaction
        [##1##2##3##4##5]
        [    PB:##4##5=>\beginchemicalpicture{##4##5},
                PE????=>\endchemicalpicture,
            %SUB##4##5=>\processchemicalsubstitute{##4##5},
             ADJ##4##5=>\processchemicaldistance{##4##5},
             MOV##4##5=>\processchemicaltranslate{##4##5},
             -SB##4##5=>\processchemicallinesegment{-SB}{##4##5},
             +SB##4##5=>\processchemicallinesegment{+SB}{##4##5},
           SB##3##4##5=>\processchemicallinesegment{SB}{##3##4##5},
         B##2##3##4##5=>\processchemicallinesegment{B}{##2##3##4##5},
         Z##2##3##4##5=>\processchemicaltextelement{Z}{##2##3##4##5}{#1}{0}{},
            \s!unknown=>\unknownchemical{##1##2##3##4##5}]}}

\def\executechemicalFIVEFRONT[#1]%
  {\executechemicalFIVE[]%
   %
   \setchemicalname FIVEFRONT
   %
   \setchemicallinesegment  -R 688 500 688 100
   \setchemicallinesegment  +R 688 500 688 900
   %
   \setchemicaltextelement -RZ   0 -1300
   \setchemicaltextelement +RZ   0  1300
   %
   \def\processchemical[##1##2##3##4]%
     {\def\chemicalrotation{2}%
      \processaction
        [##1##2##3##4]
        [    -RZ##4=>\processchemicalunrotatedtextelement{Z}{-RZ}{##4}{#1}{5}
                       {,,,,, t,t,t,t,t},
             +RZ##4=>\processchemicalunrotatedtextelement{Z}{+RZ}{##4}{#1}{5}
                       {,,,,, b,b,b,b,b},
             -SB##4=>\processchemicallinesegment{-SB}{##4},
             +SB##4=>\processchemicallinesegment{+SB}{##4},
           SB##3##4=>\processchemicallinesegment{SB}{##3##4},
           -R##3##4=>\processchemicalunrotatedlinesegment{t}{-R}{##3##4},
           +R##3##4=>\processchemicalunrotatedlinesegment{b}{+R}{##3##4},
           BB##3##4=>\processchemicalzlinesegment{B}{##3##4},
         R##2##3##4=>\processchemicalunrotatedlinesegment{t}{-R}{##2##3##4}%
                     \processchemicalunrotatedlinesegment{b}{+R}{##2##3##4},
         B##2##3##4=>\processchemicallinesegment{B}{##2##3##4},
         Z##2##3##4=>\processchemicaltextelement{Z}{##2##3##4}{#1}{0}{},
         \s!unknown=>\unknownchemical{##1##2##3##4}]}}

\def\executechemicalSIXFRONT[#1]%
  {\executechemicalSIX[]%
   %
   \setchemicalname SIXFRONT
   %
   \setchemicallinesegment -R 866 500 866 100
   \setchemicallinesegment +R 866 500 866 900
   %
   \setchemicaltextelement -RZ 0 -1300
   \setchemicaltextelement +RZ 0  1300
   %
   \def\processchemical[##1##2##3##4]%
     {\def\chemicalrotation{2}%
      \processaction
        [##1##2##3##4]
        [    -RZ##4=>\processchemicalunrotatedtextelement{Z}{-RZ}{##4}{#1}{6}
                       {,,,,,, t,t,t,t,t,t},
             +RZ##4=>\processchemicalunrotatedtextelement{Z}{+RZ}{##4}{#1}{6}
                       {,,,,,, b,b,b,b,b,b},
             -SB##4=>\processchemicallinesegment{-SB}{##4},
             +SB##4=>\processchemicallinesegment{+SB}{##4},
           SB##3##4=>\processchemicallinesegment{SB}{##3##4},
           -R##3##4=>\processchemicalunrotatedlinesegment{t}{-R}{##3##4},
           +R##3##4=>\processchemicalunrotatedlinesegment{b}{+R}{##3##4},
           BB##3##4=>\processchemicalzlinesegment{B}{##3##4},
         R##2##3##4=>\processchemicalunrotatedlinesegment{t}{-R}{##2##3##4}%
                     \processchemicalunrotatedlinesegment{b}{+R}{##2##3##4},
         B##2##3##4=>\processchemicallinesegment{B}{##2##3##4},
         Z##2##3##4=>\processchemicaltextelement{Z}{##2##3##4}{#1}{0}{},
        \s!unknown=>\unknownchemical{##1##2##3##4}]}}

% 1  :    0
% 2  : -115
% 3* : -195
% 3  : -165
% 4  : -245

\def\executechemicalCARBON[#1]%
  {\setchemicalname CARBON
   %
   \setchemicalmaximum    4
   \setchemicaldistance   0
   \setchemicalsubstitute 0
   %
   \setchemicalrotation 1   1      0      0     -1     -1      0      0      1
   \setchemicalrotation 2  -0.423 -0.906 -0.906  0.423  0.423  0.906  0.906 -0.423
   \setchemicalrotation 3  -0.966 -0.259 -0.259  0.966  0.966  0.259  0.259 -0.966
   \setchemicalrotation 3* -0.966  0.259  0.259  0.966  0.966 -0.259 -0.259 -0.966
   \setchemicalrotation 4  -0.423  0.906  0.906  0.423  0.423 -0.906 -0.906 -0.423
   %
   \setchemicalangle 1     0  90 180 270
   \setchemicalangle 2   115 205 295  25
   \setchemicalangle 3   165 255 345  75
   \setchemicalangle 3*  195 285  15 105
   \setchemicalangle 4   245 335  65 155
   %
   \setchemicaltranslate 1 -1500     0
   \setchemicaltranslate 2     0  1500
   \setchemicaltranslate 3  1500     0
   \setchemicaltranslate 4     0 -1500
   %
   \setchemicallinesegment B1   500 0  1000 0
   \setchemicallinesegment B2   300 0  1000 0
   \setchemicallinesegment B3   500 0  1000 0
   \setchemicallinesegment B4   300 0  1000 0
   %
   \setchemicaltextelement  Z  1100 0
   %
   \setchemicalcircsegment  C  0 360 500 360 0  -500
   %
   \def\processchemical[##1##2##3##4##5]%
     {\processaction
        [##1##2##3##4##5]
        [      MIR????=>\setchemicalmirror{3},
              -MIR????=>\resetchemicalmirror{3},
              *MIR????=>\togglechemicalmirror{3},
                CB????=>\processlocalchemicals{B,C,Z},
                 C????=>\processchemicalcircsegment{C}{1????},
               -ROT##5=>\reversechemical{ROT}{##5}{3,4,1,2},
             ROT##4##5=>\processchemicalrotation{##4##5},
             MOV##4##5=>\processchemicaltranslate{##4##5},
           CB##3##4##5=>\processlocalchemicals
                          {ROT##3,C,B,Z2..4,
                           MOV##3,*MIR,-ROT##3,C,B,Z2..4},
         B##2##3##4##5=>\processprivatechemicallinesegment{B}{##2##3##4##5},
         Z##2##3##4##5=>\processchemicaltextelement{Z}{##2##3##4##5}{#1}{4}
                          {l,t,r,b, t,r,b,l, r,b,l,t, b,l,t,r},
            \s!unknown=>\unknownchemical{##1##2##3##4##5}]}}

% 1:   45      2:  -90      3: -225
% 4:  -45      5: -135      6: -270

\newif\ifNEWMANstagger  \NEWMANstaggertrue

\def\executechemicalNEWMANSTAGGER%
  {\NEWMANstaggertrue\executechemicalNEWMAN}

\def\executechemicalNEWMANECLIPSE%
  {\NEWMANstaggerfalse\executechemicalNEWMAN}

\def\executechemicalNEWMAN[#1]%
  {\setchemicalname NEWMAN
   %
   \setchemicalmaximum    6
   \setchemicaldistance   0
   \setchemicalsubstitute 0
   %
   \ifNEWMANstagger
     \setchemicalrotation 1  0.707  0.707  0.707 -0.707 -0.707 -0.707 -0.707  0.707
     \setchemicalrotation 2  0     -1     -1      0      0      1      1      0
     \setchemicalrotation 3 -0.707  0.707  0.707  0.707  0.707 -0.707 -0.707 -0.707
   \else
     \setchemicalrotation 1  .866 -.5   -.5   -.866 -.866  .5    .5    .866
     \setchemicalrotation 2 -.259  .966  .966  .259  .259 -.966 -.966 -.259
     \setchemicalrotation 3 -.5   -.866 -.866  .5    .5    .866  .866 -.5
   \fi
   \setchemicalrotation 4  0.707 -0.707 -0.707 -0.707 -0.707  0.707  0.707  0.707
   \setchemicalrotation 5 -0.707 -0.707 -0.707  0.707  0.707  0.707  0.707 -0.707
   \setchemicalrotation 6  0      1      1      0      0     -1     -1      0
   %
   \ifNEWMANstagger
     \setchemicalangle 1 315  45 135 225
     \setchemicalangle 2  90 180 270   0
     \setchemicalangle 3 225 315  45 135
   \else
     \setchemicalangle 1  30 120 210 300
     \setchemicalangle 2 255 345  75 165
     \setchemicalangle 3 120 210 300  30
   \fi
   \setchemicalangle 4  45 135 225 315
   \setchemicalangle 5 135 225 315  45
   \setchemicalangle 6 270   0  90 180
   %
   \setchemicaltranslate 1 -1500     0
   \setchemicaltranslate 2     0  1500
   \setchemicaltranslate 3  1500     0
   \setchemicaltranslate 4     0 -1500
   %
   \setchemicallinesegment B1    0 0  1000 0
   \setchemicallinesegment B2    0 0  1000 0
   \setchemicallinesegment B3    0 0  1000 0
   \setchemicallinesegment B4  500 0  1000 0
   \setchemicallinesegment B5  500 0  1000 0
   \setchemicallinesegment B6  500 0  1000 0
   %
   \setchemicaltextelement  Z 1100 0
   %
   \setchemicalcircsegment  C 0 360 500 360 0 -500
   %
   \def\processchemical[##1##2##3##4]%
     {\processaction
        [##1##2##3##4]
        [STAGGER????=>{\executechemicalNEWMANSTAGGER[#1]},
         ECLIPSE????=>{\executechemicalNEWMANECLIPSE[#1]},
               B????=>\processlocalchemicals{B1..6},
              CB????=>\processlocalchemicals{B1..6,C,Z1..6},
               C????=>\processchemicalcircsegment{C}{1????},
              ROT##4=>\processchemicalrotation{##4},
              MOV##4=>\processchemicaltranslate{##4},
          B##2##3##4=>\processprivatechemicallinesegment{B}{##2##3##4},
          Z##2##3##4=>\ifNEWMANstagger
                        \processchemicaltextelement{Z}{##2##3##4}{#1}{6}
                          {l,t,r,l,r,b, l,r,l,r,r,l, r,b,l,r,l,t, r,l,r,l,l,r}%
                      \else
                        \processchemicaltextelement{Z}{##2##3##4}{#1}{6}
                          {l,r,t,t,r,b, t,b,r,r,b,l, r,l,b,b,l,t, b,t,l,l,t,r}%
                      \fi,
          \s!unknown=>\unknownchemical{##1##2##3##4}]}}

\def\executechemicalCHAIR[#1]% smaller
  {\setchemicalname CHAIR
   %
   \setchemicalmaximum 6
   %
   \setchemicallinesegment  B1  1600  800  2800 -800
   \setchemicallinesegment  B2  2800 -800   800    0
   \setchemicallinesegment  B3   800    0 -1600 -800
   \setchemicallinesegment  B4 -1600 -800 -2800  800
   \setchemicallinesegment  B5 -2800  800  -800    0
   \setchemicallinesegment  B6  -800    0  1600  800
   %
   \setchemicallinesegment +R1  1600  800  1600  1600
   \setchemicallinesegment +R2  2800 -800  2800 -1600
   \setchemicallinesegment +R3   800    0   800   800
   \setchemicallinesegment +R4 -1600 -800 -1600 -1600
   \setchemicallinesegment +R5 -2800  800 -2800  1600
   \setchemicallinesegment +R6  -800    0  -800  -800
   %
   \setchemicallinesegment -R1  1600  800  2350  522 % 750 278
   \setchemicallinesegment -R2  2800 -800  3493 -400
   \setchemicallinesegment -R3   800    0  1329 -600 % 528 600
   \setchemicallinesegment -R4 -1600 -800 -2350 -522 % 750 278
   \setchemicallinesegment -R5 -2800  800 -3493  400
   \setchemicallinesegment -R6  -800    0 -1329  600 % 528 600
   %
   \setchemicaltextelement +RZ1  1600  1800
   \setchemicaltextelement +RZ2  2800 -1800
   \setchemicaltextelement +RZ3   800  1000
   \setchemicaltextelement +RZ4 -1600 -1800
   \setchemicaltextelement +RZ5 -2800  1800
   \setchemicaltextelement +RZ6  -800 -1000
   %
   \setchemicaltextelement -RZ1  2538  453 % 200 lang
   \setchemicaltextelement -RZ2  3666 -300
   \setchemicaltextelement -RZ3  1460 -750
   \setchemicaltextelement -RZ4 -2538 -453
   \setchemicaltextelement -RZ5 -3666  300
   \setchemicaltextelement -RZ6 -1460  750
   %
   \def\processchemical[##1##2##3##4##5]%
     {\def\chemicalrotation{1}%
      \processaction
        [##1##2##3##4##5]
        [      B????=>\processlocalchemicals{B1,B2,B3,B4,B5,B6},
              -R????=>\processlocalchemicals{-R1,-R2,-R3,-R4,-R5,-R6},
              +R????=>\processlocalchemicals{+R1,+R2,+R3,+R4,+R5,+R6},
            B##2????=>{\getchemicallinesegment[0][B##2]},
          -RZ##4????=>{\getchemicalfixedtextelement[-RZ##4][1][##4][l,l,tc,r,r,bc][#1]},
          +RZ##4????=>{\getchemicalfixedtextelement[+RZ##4][1][##4][c][#1]},
           -R##3????=>{\getchemicallinesegment[0][-R##3]},
           +R##3????=>{\getchemicallinesegment[0][+R##3]},
          \s!unknown=>\unknownchemical{##1##2##3##4##5}]}}

\def\executechemicalarrow#1#2[#3]%
  {\dogetcommalistelement1\from#3\to\toptext
   \dogetcommalistelement2\from#3\to\bottext
   \def\dochemicaltext##1%
     {\dosetsubscripts%
      $\@@chemicalletter{\@@localchemicalformat\strut##1}$%
      \doresetsubscripts}%
   \doifelse{\@@chemicalplaats}{\v!intekst}%
     {#1{\dochemicaltext\toptext}}%
     {\setbox\chemicalsymbols=\hbox
        {\box\chemicalsymbols
         \vbox{\halign{##\cr
           \hbox to 3em{\hss\dochemicaltext{\toptext}\hss}\cr
           #2%
           \hbox to 3em{\hss\dochemicaltext{\bottext}\hss}\cr}}}}}

\def\executechemicalGIVES%
  {\executechemicalarrow
     {\chemicalsinglepicturearrow}% nodig
     {\rightarrowfill\cr}}

\def\executechemicalEQUILIBRIUM%
  {\executechemicalarrow
     {\chemicaldoublepicturearrow}%  nodig
     {\rightarrowfill\cr\leftarrowfill\cr}}

\def\executechemicalMESOMERIC%
  {\executechemicalarrow
     {\chemicalsinglepicturearrow}%  nodig
     {$\leftarrow\hskip-1em$\rightarrowfill\cr}}

\def\executechemicalsign#1[#2]%
  {\doifelse{\@@chemicalplaats}{\v!intekst}
     {\dosetsubscripts
      $\@@chemicalletter{\@@localchemicalformat#1}$%
      \doresetsubscripts}
     {\setbox\chemicalsymbols=\hbox
        {\box\chemicalsymbols
         \dosetsubscripts
         $\@@chemicalletter{\@@localchemicalformat#1}$%
         \doresetsubscripts}}}

\def\executechemicalPLUS%
  {\executechemicalsign{+}}

\def\executechemicalMINUS%
  {\executechemicalsign{-}}

\def\executechemicalEQUAL%
  {\executechemicalsign{=}}

\def\executechemicalSPACE[#1]%
  {\doifnot{\@@chemicalplaats}{\v!intekst}%
     {\setbox\chemicalsymbols=\hbox
        {\box\chemicalsymbols
         \quad}}}

\def\executechemicalCHEM[#1]%
  {\doifnot{\@@chemicalplaats}{\v!intekst}%
     {\setbox\chemicalsymbols=\hbox
        {\box\chemicalsymbols
         $\@@chemicalletter{\@@localchemicalformat#1}$}}}

\def\executechemicalTEXT[#1]%
  {\doifnot{\@@chemicalplaats}{\v!intekst}%
     {\setbox\chemicalsymbols=\hbox
        {\box\chemicalsymbols#1}}}

%\def\executechemicalLOW[#1]%
%  {\setlowsubscripts}
%
%\def\executechemicalHIGH[#1]%
%  {\sethighsubscripts}

\def\putchemicalrule#1#2#3#4%
  {\ifcase\chemicaldrawingmode
     \putrule from {#1} {#2} to {#3} {#4}
   \or
     \psline(#1,#2)(#3,#4)%
   \or 
     \bgroup
     \!!counta=#1\!!countb=#2\!!countc=#3\!!countd=#4\relax
     \global\MPdrawingdonetrue
     \startMPdrawing
       x1 := \MPdivten[\the\!!counta]u ;
       y1 := \MPdivten[\the\!!countb]u ;
       x2 := \MPdivten[\the\!!countc]u ;
       y2 := \MPdivten[\the\!!countd]u ;
       draw z1--z2 ;
     \stopMPdrawing
     \egroup
   \fi}

\def\executechemicalcomplex#1%
  {\bgroup
   \putchemicalrule {0} {-\@@chemicalonder}     {0}  {\@@chemicalboven}%
   \putchemicalrule {0}  {\@@chemicalboven} {#1150}  {\@@chemicalboven}%
   \putchemicalrule {0} {-\@@chemicalonder} {#1150} {-\@@chemicalonder}%
   \egroup}

\def\executechemicalOPENCOMPLEX[#1]%
  {\executechemicalcomplex+\ignorespaces
   \executechemicalSPACE[]}

\def\executechemicalCLOSECOMPLEX[#1]%
  {\executechemicalSPACE[]%
   \executechemicalcomplex-\ignorespaces}

% nog niet door midden as!

\def\executechemicalverticalsymbol#1#2%
  {\executechemicalTEXT
     [$\left#1\relax
       \dimen0=\@@chemicalunit
       \scratchcounter=\@@chemicalboven
       \advance\scratchcounter by \@@chemicalonder
       \dimen0=\scratchcounter\dimen0
       \vcenter to \dimen0{}
       \dimen2=\@@chemicalunit
       \dimen2=\@@chemicalrechts\dimen0  
       \vcenter{\leftskip1em\hsize\dimen2\relax\strut#2\strut}%
       \right.$]}%

\def\executechemicalUPARROW[#1]%
  {\executechemicalverticalsymbol\uparrow{#1}}

\def\executechemicalDOWNARROW[#1]%
  {\executechemicalverticalsymbol\downarrow{#1}}

\def\executechemicalUPDOWNARROW[#1]%
  {\executechemicalverticalsymbol\updownarrow{#1}}

\setupchemical
  [\c!breedte=0,
   \c!hoogte=0,
   \c!links=0,
   \c!rechts=0,
   \c!boven=0,
   \c!onder=0,
   \c!korps=\the\bodyfontsize,
   \c!resolutie=\outputresolution,
   \c!schaal=\v!middel,
   \c!formaat=\v!middel,
   \c!tekstformaat=\v!groot,
   \c!kader=\v!uit,
   \c!assenstelsel=\v!uit,
   \c!status=\v!start,
   \c!letter=\rm,
   \c!plaats=,
   \c!optie=,
   \c!offset=LOW,
   \c!variant=1]

% Tijdelijk plaatsen we deze extra macro's hier.
%
% mathontop: \mtop {} {}
% textontop: \ttop {} {}

\def\putontop#1#2%
  {\vbox
     {\halign
        {\strut\hss##\hss\cr
         #1\cr
         #2\cr}}}

\def\ttop#1#2%
  {\putontop{\tx#1}{#2}}

\def\mtop#1#2%
  {\vbox
     {\offinterlineskip
      \halign
        {\hss##\hss\cr
         $\scriptscriptstyle#1$\cr
         \noalign{\vskip.5ex}%
         $#2$\cr}}}

\def\ctop#1#2%
  {\vbox
     {\offinterlineskip
      \halign
        {\hss##\hss\cr
         $\@@chemicalletter{\@@localchemicalformat\scriptscriptstyle#1}$\cr
         \noalign{\vskip.5ex}%
         $\@@chemicalletter{\@@localchemicalformat#2}$\cr}}}

\protect

\endinput
